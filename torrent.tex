% \begin{document}

\usetikzlibrary{arrows.meta} % For double arrows

\chapter{BitTorrent Protocol}

BitTorrent is a peer-to-peer file sharing protocol that allows nodes to 
distribute data in a de-centralized fashion. The tracker is the central server
that knows all the members in the cluster. A node coming online learns about its
peers by talking to the tracker. Once a node learned about the peer nodes, the
node can initiate data exchange with its peers directly. Files are exchanged in 
chunks, and a node will contact its peers to figure out who has data it doesn't
have.

\pagebreak

The following is a BitTorrent cluster with five nodes:

\begin{center}
\begin{tikzpicture}
    % Draw 5 "client" nodes in a bigger circle, labeled from 1 to 5
    \foreach \angle/\label in {0/1, 72/2, 144/3, 216/4, 288/5} {
        \node[draw, circle] (client\angle) at (\angle:5cm) {Node \label}; % Increased radius to 5cm
    }

    % Draw the "tracker" node in the center
    \node[draw, circle] (tracker) at (0,0) {Tracker};

    % Draw double-ended dotted lines from each client to the tracker
    \foreach \angle in {0, 72, 144, 216, 288} {
        \draw[Latex-Latex, dotted] (client\angle) -- (tracker);
    }

    % Draw single-line, double-ended arrows between all client nodes
    \foreach \source in {0, 72, 144, 216, 288} {
        \foreach \target in {72, 144, 216, 288, 0} {
            \ifnum\source<\target % Avoid duplicate and self-loops
                \draw[Latex-Latex] (client\source) -- (client\target);
            \fi
        }
    }
\end{tikzpicture}
\end{center}

\section{Design}

In this chapter we will implement a simple BitTorrent cluster. The cluster will
have a single tracker and N nodes. A node can join or leave the cluster, and
make progress while inside the cluster.\\

To simplify the model, we assume a single file with N chunks is shared within
the cluster. We also assume at any moment the cluster contains the entire file.
This means while a node can leave the cluster at any time, it can only leave if
the cluster still have the entire file after it leaves.

% \pagebreak

\section{Spec}

The cluster starts with a single \textit{Seed} node that has the entire data
set. In every step, a node can \textit{Join}, \textit{Progress}, or
\textit{Leave}.\\

\begin{tla}
Init ==
    /\ tracker = {Seed}
    /\ data = [k \in Client |-> IF k = Seed THEN AllChunks ELSE {}]

Next ==
    \/ \E k \in Client: 
        Join(k) 
    \/ \E k \in Client: 
        Progress(k)
    \/ \E k \in Client: 
        Leave(k)
\end{tla}
\begin{tlatex}
\@x{ Init \.{\defeq}}%
\@x{\@s{16.4} \.{\land} tracker \.{=} \{ Seed \}}%
 \@x{\@s{16.4} \.{\land} data \.{=} [ k \.{\in} Client \.{\mapsto} {\IF} k
 \.{=} Seed \.{\THEN} AllChunks \.{\ELSE} \{ \} ]}%
\@pvspace{8.0pt}%
\@x{ Next \.{\defeq}}%
\@x{\@s{16.4} \.{\lor} \E\, k \.{\in} Client \.{:}}%
\@x{\@s{20.5} Join ( k )}%
\@x{\@s{16.4} \.{\lor} \E\, k \.{\in} Client \.{:}}%
\@x{\@s{20.5} Progress ( k )}%
\@x{\@s{16.4} \.{\lor} \E\, k \.{\in} Client \.{:}}%
\@x{\@s{20.5} Leave ( k )}%
\end{tlatex}
\\

A node can join a cluster if it's not currently part of it:\\
\begin{tla}
Join(k) == 
    /\ k \notin tracker
    /\ tracker' = tracker \cup {k}
    /\ UNCHANGED data
\end{tla}
\begin{tlatex}
\@x{ Join ( k ) \.{\defeq}}%
\@x{ \.{\land} k \.{\notin} tracker}%
\@x{ \.{\land} tracker \.{'} \.{=} tracker \.{\cup} \{ k \}}%
\@x{ \.{\land} {\UNCHANGED} data}%
\end{tlatex}
\\

A node can leave a cluster. To simplify the model, a node k can leave if and
only if the cluster still have the full data set without k. Node a node can
\textit{Leave} even if it doesn't have the whole set of data:\\

\begin{tla}
AllDataWithout(k) == 
    UNION {data[i] : i \in tracker \ {k}}

Leave(k) == 
    /\ k \in tracker
    /\ AllDataWithout(k) = AllChunks
    /\ tracker' = tracker \ {k}
    /\ data' = [data EXCEPT ![k] = {}] 
\end{tla}
\begin{tlatex}
\@x{ AllDataWithout ( k ) \.{\defeq}}%
 \@x{\@s{16.4} {\UNION} \{ data [ i ] \.{:} i \.{\in} tracker
 \.{\,\backslash\,} \{ k \} \}}%
\@pvspace{8.0pt}%
\@x{ Leave ( k ) \.{\defeq}}%
\@x{\@s{16.4} \.{\land} k \.{\in} tracker}%
\@x{\@s{16.4} \.{\land} AllDataWithout ( k ) \.{=} AllChunks}%
 \@x{\@s{16.4} \.{\land} tracker \.{'} \.{=} tracker \.{\,\backslash\,} \{ k
 \}}%
 \@x{\@s{16.4} \.{\land} data \.{'} \.{=} [ data {\EXCEPT} {\bang} [ k ] \.{=}
 \{ \} ]}%
\end{tlatex}
\\

Finally, for a node U to make progress: U must have incomplete set of data and
there's another node V that has data U doesn't have:\\

\begin{tla}
Progress(u) == 
    /\ u \in tracker
    /\ data[u] # AllChunks  \* u is incomplete
    /\ \E v \in tracker:
        \E k \in AllChunks:
            /\ k \notin data[u]     \* v has something u doesn't
            /\ k \in data[v] 
            /\ data' = [data EXCEPT ![u] = data[u] \cup {k}]
            /\ UNCHANGED tracker
\end{tla}
\begin{tlatex}
\@x{ Progress ( u ) \.{\defeq}}%
\@x{\@s{16.4} \.{\land} u \.{\in} tracker}%
\@x{\@s{16.4} \.{\land} data [ u ] \.{\neq} AllChunks\@s{4.1}}%
\@y{%
  u is incomplete
}%
\@xx{}%
\@x{\@s{16.4} \.{\land} \E\, v \.{\in} tracker \.{:}}%
\@x{\@s{20.5} \E\, k \.{\in} AllChunks \.{:}}%
\@x{\@s{24.6} \.{\land} k \.{\notin} data [ u ]\@s{16.4}}%
\@y{%
  v has something u doesn't
}%
\@xx{}%
\@x{\@s{24.6} \.{\land} k \.{\in} data [ v ]}%
 \@x{\@s{24.6} \.{\land} data \.{'} \.{=} [ data {\EXCEPT} {\bang} [ u ] \.{=}
 data [ u ] \.{\cup} \{ k \} ]}%
\@x{\@s{24.6} \.{\land} {\UNCHANGED} tracker}%
\end{tlatex}

\section{Safety}

A requirement we imposed on the design is to ensure the cluster contain the
entire file at all times:\\

\begin{tla}
Safety == 
    UNION {data[k] : k \in Client} = AllChunks
\end{tla}
\begin{tlatex}
\@x{ Safety \.{\defeq}}%
 \@x{\@s{16.4} {\UNION} \{ data [ k ] \.{:} k \.{\in} Client \} \.{=}
 AllChunks}%
\end{tlatex}

\section{Liveness}

We want to verify a newly joined node eventually gets the entire file. However, 
verifying all nodes eventually get the entire file took too long. We will define
check liveness property for only one node:\\

\begin{tla}
NodeToVerify == "c0"

Liveness == 
    \* targeted liveness condition
    data[NodeToVerify] = {} ~> data[NodeToVerify] = AllChunks

Spec ==
    /\ Init
    /\ [][Next]_vars
    /\ WF_vars(Next)
    \* targeted fairness description
    /\ SF_vars(Join(NodeToVerify))
    /\ \A s \in SUBSET AllChunks: 
        SF_vars(data[NodeToVerify] = s /\ Progress(NodeToVerify))
\end{tla}
\begin{tlatex}
\@x{ NodeToVerify \.{\defeq}\@w{c0}}%
\@pvspace{8.0pt}%
\@x{ Liveness \.{\defeq}}%
\@x{\@s{16.4}}%
\@y{%
  targeted liveness condition
}%
\@xx{}%
 \@x{\@s{16.4} data [ NodeToVerify ] \.{=} \{ \} \.{\leadsto} data [
 NodeToVerify ] \.{=} AllChunks}%
\@pvspace{8.0pt}%
\@x{ Spec \.{\defeq}}%
\@x{\@s{16.4} \.{\land} Init}%
\@x{\@s{16.4} \.{\land} {\Box} [ Next ]_{ vars}}%
\@x{\@s{16.4} \.{\land} {\WF}_{ vars} ( Next )}%
\@x{\@s{16.4}}%
\@y{%
  targeted fairness description
}%
\@xx{}%
\@x{\@s{16.4} \.{\land} {\SF}_{ vars} ( Join ( NodeToVerify ) )}%
\@x{\@s{16.4} \.{\land} \A\, s \.{\in} {\SUBSET} AllChunks \.{:}}%
 \@x{\@s{20.5} {\SF}_{ vars} ( data [ NodeToVerify ] \.{=} s \.{\land}
 Progress ( NodeToVerify ) )}%
\end{tlatex}
\\

The liveness property checks for a node with no data eventually gets all the
data. \\

However, without fairness, \textit{Spec} permits a node to \textit{never} make
progress by having it repeatedly joining and leaving the cluster. To ensure 
\textit{NodeToVerify} always make progress, we need to enable strong fairness 
for \textit{Progress(NodeToVerify)}. However, this alone is not enough. We need 
to ensure \textit{Progress(NodeToVerify)} is called for \textit{all} possible
\textit{NodeToVerify} state, this is solved using the \textit{SUBSET} keyword.

% \end{document}
