\chapter{General Tips}

\section{Model Checker Debug}

Debugging in TLC is a bit different than debugging with normal programs. A step
in the model checker is really a state transition. Even if the model cheker
completes, it's still worthwhile dump and audit the states just to make sure the
Spec is defined correctly.

\begin{verbatim}
tlc elevator -dump out > /dev/null && cat out.dump | head -n5
State 1:
a = 1
State 2:
a = 2
\end{verbatim}

You may want to grep the output to look for state being set to certain value to 
confirm the Spec is working as intended.

\subsection{Dead Lock}

Deadlock typically happens when the model checker ran out of things to do. This
is typically a result of an incomplete Spec definition, where certain edge cases
were not accounted for. The model checker typically provides a fairly
comprehensive backtrace leading up to the dead lock to simplify debug.

\subsection{Live Lock}

Livelock happens when the model checker identifies a case where the liveness
property is violated. An example is the elevator stuck going between two floors 
instead of keep going to the top floor, or the system is stuck dropping and
retransmit the same packet.\newline 

These are typically fixed by providing additional fairness description to the
Spec, telling the model checker how continue in the case of a live lock.\newline 

For a detail fairness description please refer to Chapter~\ref{chap:fairness}.

\section{Model Refinement}

This is, in some sense, the \textit{art} associated with writing model checker
verifiable TLA+ spec. Model checking is only valuable if it can be verified
within a reasonable amount of time. Since the model complexity grows
exponentially, there's little value in attempting to hyper-optimize the details.
Designer should focus on optimizing the broad stroke, such as removing features 
that are harder to get wrong, and focus the model on the bits that have the
highest return on investment.\newline

One useful way of trimming out low value portion of the Spec is to audit the
state dump. Even in the case of a non-terminating run, a partial state dump can
help identify low value details we can remove from the Spec.\newline

One key value of TLA+ is it highlights all the corner cases in the system. Even
if you end up removing certain details from the Spec, it still likely highlights
to you certain condition you were unaware of.\newline

As a broad stroke principle: when the Spec has millions or higher more states,
it is unlikely to terminate within a few seconds. From first principles if you
can find one failure case in a million states, you can also likely reduce the 
scope of the model to reproduce the failure in much fewer states.