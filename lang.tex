
% \begin{document}

\chapter{Data Structure}

Like other languages, TLA+ provides its data structure. Readers are assumed
familiar with common data structures, and this chapter will only focus on the
TLA+ language semantics.

\section{Set}

Set is an unordered set where every element in the set is unique. TLA+ Set
includes common set operation including union, intersection, membership check,
and more.\newline

\begin{tla}
a == {0, 1, 2}
b == {2, 3, 4}

\* \{0, 1, 2, 3, 4\}
c == a \union b         

\* \{2\}
d == a \intersect b     

\* TRUE - because 4 in c is bigger than 3
e == \E x \in c: x > 3  

\* FALSE - nothing in c is bigger than 5
f == \E x \in c: x > 5  

\* FALSE - not all elements in c are smaller than 3
g == \A x \in c: x < 3  

\* TRUE - all elements in c are smaller than 3
h == \A x \in c: x < 5  

\* \{0, 1, 2\} - all elementse less than 3
i == {x \in c: x < 3}   

\* 5 - the number of elements in c
j == Cardinality(c)     

\* \{0, 1, 3, 4\} - c substracts d
k == c \ d             
\end{tla}
\begin{tlatex}
\@x{ a \.{\defeq} \{ 0 ,\, 1 ,\, 2 \}}%
\@x{ b \.{\defeq} \{ 2 ,\, 3 ,\, 4 \}}%
\@pvspace{8.0pt}%
\@x{}%
\@y{%
  \{0, 1, 2, 3, 4\}
}%
\@xx{}%
\@x{ c \.{\defeq} a \.{\cup} b}%
\@pvspace{8.0pt}%
\@x{}%
\@y{%
  \{2\}
}%
\@xx{}%
\@x{ d \.{\defeq} a \.{\cap} b}%
\@pvspace{8.0pt}%
\@x{}%
\@y{%
  TRUE - because 4 in c is bigger than 3
}%
\@xx{}%
\@x{ e \.{\defeq} \E\, x \.{\in} c \.{:} x \.{>} 3}%
\@pvspace{8.0pt}%
\@x{}%
\@y{%
  FALSE - nothing in c is bigger than 5
}%
\@xx{}%
\@x{ f \.{\defeq} \E\, x \.{\in} c \.{:} x \.{>} 5}%
\@pvspace{8.0pt}%
\@x{}%
\@y{%
  FALSE - not all elements in c are smaller than 3
}%
\@xx{}%
\@x{ g \.{\defeq} \A\, x \.{\in} c \.{:} x \.{<} 3}%
\@pvspace{8.0pt}%
\@x{}%
\@y{%
  TRUE - all elements in c are smaller than 3
}%
\@xx{}%
\@x{ h \.{\defeq} \A\, x \.{\in} c \.{:} x \.{<} 5}%
\@pvspace{8.0pt}%
\@x{}%
\@y{%
  \{0, 1, 2\} - all elementse less than 3
}%
\@xx{}%
\@x{ i \.{\defeq} \{ x \.{\in} c \.{:} x \.{<} 3 \}}%
\@pvspace{8.0pt}%
\@x{}%
\@y{%
  5 - the number of elements in c
}%
\@xx{}%
\@x{ j \.{\defeq} Cardinality ( c )}%
\@pvspace{8.0pt}%
\@x{}%
\@y{%
  \{0, 1, 3, 4\} - c substracts d
}%
\@xx{}%
\@x{ k \.{\defeq} c \.{\,\backslash\,} d}%
\end{tlatex}

\section{Tuple}

A tuple is an ordered data structure, similar to a queue in other languages.
Common operation supported by tuple include Append to push and Tail to pop.
\newline

\begin{tla}
a == <<0, 1, 2>>                    
b == <<2, 3, 4>>

\* tuple: 0, 1, 2, 2, 3, 4
c == A \o B

\* tuple: 0, 1, 2, 3
c2 == Append(A, 3) 

\* gets head: 0
c3 == Head(A) 

\* removes head: 1, 2
c4 == Tail(A) 

\* 6
d == Len(c)                         

\* TRUE - every c[x] is not 10
\* First tuple element is at index 1 (not 0)
e == \A x \in 1..Len(c) : c[x] # 10 

\* TRUE - there exists a c[x] that is 2
f == \E x \in 1..Len(c) : c[x] = 2

\* \{3, 4\} - when index is 3 or 4, c[x] = 2
g == {x \in 1..Len(c) : c[x] = 2}   
\end{tla}
\begin{tlatex}
\@x{ a \.{\defeq} {\langle} 0 ,\, 1 ,\, 2 {\rangle}}%
\@x{ b \.{\defeq} {\langle} 2 ,\, 3 ,\, 4 {\rangle}}%
\@pvspace{8.0pt}%
\@x{}%
\@y{%
  tuple: 0, 1, 2, 2, 3, 4
}%
\@xx{}%
\@x{ c \.{\defeq} A \.{\circ} B}%
\@pvspace{8.0pt}%
\@x{}%
\@y{%
  tuple: 0, 1, 2, 3
}%
\@xx{}%
\@x{ c2 \.{\defeq} Append ( A ,\, 3 )}%
\@pvspace{8.0pt}%
\@x{}%
\@y{%
  gets head: 0
}%
\@xx{}%
\@x{ c3 \.{\defeq} Head ( A )}%
\@pvspace{8.0pt}%
\@x{}%
\@y{%
  removes head: 1, 2
}%
\@xx{}%
\@x{ c4 \.{\defeq} Tail ( A )}%
\@pvspace{8.0pt}%
\@x{}%
\@y{%
  6
}%
\@xx{}%
\@x{ d \.{\defeq} Len ( c )}%
\@pvspace{8.0pt}%
\@x{}%
\@y{%
  TRUE - every c[x] is not 10
}%
\@xx{}%
\@x{}%
\@y{%
  First tuple element is at index 1 (not 0)
}%
\@xx{}%
 \@x{ e \.{\defeq} \A\, x \.{\in} 1 \.{\dotdot} Len ( c ) \.{:} c [ x ]
 \.{\neq} 10}%
\@pvspace{8.0pt}%
\@x{}%
\@y{%
  TRUE - there exists a c[x] that is 2
}%
\@xx{}%
 \@x{ f \.{\defeq} \E\, x \.{\in} 1 \.{\dotdot} Len ( c ) \.{:} c [ x ] \.{=}
 2}%
\@pvspace{8.0pt}%
\@x{}%
\@y{%
  \{3, 4\} - when index is 3 or 4, c[x] = 2
}%
\@xx{}%
 \@x{ g \.{\defeq} \{ x \.{\in} 1 \.{\dotdot} Len ( c ) \.{:} c [ x ] \.{=} 2
 \}}%
\end{tlatex}

\section{Function}

Function is similar to map in other data structures, supporting key value
lookup.\\

\begin{tla}
SetA == {"a", "b", "c"}
SetB == {"c", "d", "e"}

\* Create a mapping with keys a, b, c with values 0, 0, 0
a == [k \in SetA |-> 0]
b == [k \in SetB |-> 1]

\* Concatenate 
c == a @@ b

\* Subtraction
d == [x \in (DOMAIN c \ DOMAIN b) |-> c[x]]

\* Create a mapping with keys a, b, c with values \{\}, \{\}, \{\}
e == [k \in SetA |-> {}]

\* Create a mapping that is the same as e, except key a's value is {"a", "b", "c"}
f == [e EXCEPT !["a"] = {"a", "b", "c"}] 

\end{tla}
\begin{tlatex}
\@x{ SetA \.{\defeq} \{\@w{a} ,\,\@w{b} ,\,\@w{c} \}}%
\@x{ SetB \.{\defeq} \{\@w{c} ,\,\@w{d} ,\,\@w{e} \}}%
\@pvspace{8.0pt}%
\@x{}%
\@y{%
  Create a mapping with keys a, b, c with values 0, 0, 0
}%
\@xx{}%
\@x{ a \.{\defeq} [ k \.{\in} SetA \.{\mapsto} 0 ]}%
\@x{ b \.{\defeq} [ k \.{\in} SetB \.{\mapsto} 1 ]}%
\@pvspace{8.0pt}%
\@x{}%
\@y{%
  Concatenate 
}%
\@xx{}%
\@x{ c \.{\defeq} a \.{\,@@\,} b}%
\@pvspace{8.0pt}%
\@x{}%
\@y{%
  Subtraction
}%
\@xx{}%
 \@x{ d \.{\defeq} [ x \.{\in} ( {\DOMAIN} c \.{\,\backslash\,} {\DOMAIN} b )
 \.{\mapsto} c [ x ] ]}%
\@pvspace{8.0pt}%
\@x{}%
\@y{%
  Create a mapping with keys a, b, c with values \{\}, \{\}, \{\}
}%
\@xx{}%
\@x{ e \.{\defeq} [ k \.{\in} SetA \.{\mapsto} \{ \} ]}%
\@pvspace{8.0pt}%
\@x{}%
\@y{%
 Create a mapping that is the same as e, except key a's value is {"a", "b",
 "c"}
}%
\@xx{}%
 \@x{ f \.{\defeq} [ e {\EXCEPT} {\bang} [\@w{a} ] \.{=} \{\@w{a} ,\,\@w{b}
 ,\,\@w{c} \} ]}%
\@pvspace{8.0pt}%
\end{tlatex}
