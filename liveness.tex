% \begin{document}

\chapter{Liveness}
\label{chap:liveness}

While safety properties can catch per-state contradictions, liveness properties
allow you to verify the behavior across a series of states. This is TLA+'s
\textit{superpower}. We are rarely interested only in the correctness of one
state in the system, but rather in the correctness of system behavior \textit{across} a set of states.\\

This book has already provided a few examples of liveness properties: eg.
The elevator eventually makes it to the top floor, consensus protocol eventually
converges, the scheduling algorithm guarantees a lock requester eventually gets the
lock, etc. I argue any system worth the reader's time to model using TLA+ must 
have interesting liveness properties to verify.\\

Unfortunately, liveness check also takes \textit{much} longer, since the very
definition of verifying property across a series of states makes the task very
hard to parallelize. Care must go into refining the model to keep the model
checker runtime reasonable. In this chapter, we will go through a very simple
state machine to demonstrate liveness properties. \\

Assume a simple three-system system:\\

\begin{center}
\begin{tikzpicture}[>=stealth',shorten >=1pt,auto,node distance=2cm]
  \node[state]  (q1)                {1};
  \node[state]  (q2) [right of=q1]  {2};
  \node[state]  (q3) [right of=q2]  {3};

  \path[->]          (q1)  edge   []   node {} (q2);
%   \path[->]          (q2)  edge   [bend left=20]   node {} (q1);

  \path[->]          (q2)  edge   [bend left=20]   node {} (q3);
  \path[->]          (q3)  edge   [bend left=20]   node {} (q2);
\end{tikzpicture}
\end{center}

This can be described by the following spec:\newline

\begin{tla}
--------------------------- MODULE liveness ----------------------------
EXTENDS Naturals
VARIABLES counter 
vars == <<counter>>

EventuallyAlways == <>[](counter = 3)
AlwaysEventually == []<>(counter = 3)

Init ==
    /\ counter = 0

Inc == 
    /\ counter' = counter + 1

Dec == 
    /\ counter' = counter - 1

Next ==
    \/ /\ counter # 3
       /\ Inc
    \/ /\ counter = 3
       /\ Dec

Spec ==
  /\ Init
  /\ [][Next]_vars
  /\ WF_vars(Next)
=============================================================================
\end{tla}
\begin{tlatex}
\@x{}\moduleLeftDash\@xx{ {\MODULE} liveness}\moduleRightDash\@xx{}%
\@x{ {\EXTENDS} Naturals}%
\@x{ {\VARIABLES} counter}%
\@x{ vars \.{\defeq} {\langle} counter {\rangle}}%
\@pvspace{8.0pt}%
\@x{ EventuallyAlways \.{\defeq} {\Diamond} {\Box} ( counter \.{=} 3 )}%
\@x{ AlwaysEventually \.{\defeq} {\Box} {\Diamond} ( counter \.{=} 3 )}%
\@pvspace{8.0pt}%
\@x{ Init \.{\defeq}}%
\@x{\@s{16.4} \.{\land} counter \.{=} 0}%
\@pvspace{8.0pt}%
\@x{ Inc \.{\defeq}}%
\@x{ \.{\land} counter \.{'} \.{=} counter \.{+} 1}%
\@pvspace{8.0pt}%
\@x{ Dec \.{\defeq}}%
\@x{ \.{\land} counter \.{'} \.{=} counter \.{-} 1}%
\@pvspace{8.0pt}%
\@x{ Next \.{\defeq}}%
\@x{\@s{16.4} \.{\lor} \.{\land} counter \.{\neq} 3}%
\@x{\@s{16.4} \.{\land} Inc}%
\@x{\@s{16.4} \.{\lor} \.{\land} counter \.{=} 3}%
\@x{\@s{16.4} \.{\land} Dec}%
\@pvspace{8.0pt}%
\@x{ Spec \.{\defeq}}%
\@x{\@s{8.2} \.{\land} Init}%
\@x{\@s{8.2} \.{\land} {\Box} [ Next ]_{ vars}}%
\@x{\@s{8.2} \.{\land} {\WF}_{ vars} ( Next )}%
\@x{}\bottombar\@xx{}%
\end{tlatex}

Note the fairness description in the spec. Without fairness the spec is
allowed to stutter and fail any liveness property checks.

\section{Always Eventually}

We want to verify the system \textit{always eventually} transition state 3. This
can be described by the following liveness property:\newline

\begin{tla}
    AlwaysEventually == []<>(counter = 3)
\end{tla}
\begin{tlatex}
 \@x{\@s{16.4} AlwaysEventually \.{\defeq} {\Box} {\Diamond} ( counter \.{=} 3
 )}%
\end{tlatex}
\newline

Once the system makes it to state 3, the system is stuck in a loop transitioning
state 2 and 3. It doesn't \textit{remain} in state 3, but it does \textit{always
eventually} transition to state 3. The system as described fulfills this liveness
property.

\section{Eventually Always}

However, the system does not \textit{eventually always} remain in state 3, because
the system toggles between state 2 and 3. The liveness property to check the
system \textit{eventually always} transition to and remain state 3 is shown
below:\\

\begin{tla}
    EventuallyAlways == <>[](counter = 3)
\end{tla}
\begin{tlatex}
 \@x{\@s{16.4} EventuallyAlways \.{\defeq} {\Diamond} {\Box} ( counter \.{=} 3
 )}%
\end{tlatex}
\newline

To satisfy this liveness property, we will need to \textit{remove} the
transition from 3 to 2, which updates the state diagram like below:

\begin{center}
\begin{tikzpicture}[>=stealth',shorten >=1pt,auto,node distance=2cm]
  \node[state]  (q1)                {1};
  \node[state]  (q2) [right of=q1]  {2};
  \node[state]  (q3) [right of=q2]  {3};

  \path[->]          (q1)  edge   []   node {} (q2);
%   \path[->]          (q2)  edge   [bend left=20]   node {} (q1);

  \path[->]          (q2)  edge   []   node {} (q3);
%   \path[->]          (q3)  edge   [bend left=20]   node {} (q2);
\end{tikzpicture}
\end{center}

We need to remove the corresponding \textit{Dec} action from Next:\\
\begin{tla}
Next ==
    /\ counter # 3
    /\ Inc
\end{tla}
\begin{tlatex}
\@x{ Next \.{\defeq}}%
\@x{\@s{16.4} \.{\lor} \.{\land} counter \.{\neq} 3}%
\@x{\@s{16.4} \.{\land} Inc}%
\end{tlatex}
\newline 

The system now \textit{eventually always} remain in state 3, satisfy the 
liveness property.\newline 

Note that the system still \textit{always eventually} make it to state 3, so the
updated spec satisfies both \textit{AlwaysEventually} and
\textit{EventuallyAlways} liveness properties. This is not to say designer
should always use \textit{eventually always}. Some system may \textit{never}
converge onto a fixed state. For example, in a consesus system any given server
may crash and disturb the converged state. In such case \textit{eventually
always} will never be true, but \textit{always eventually} can be true.

\section{Leads To}

\textit{Leads to} provides a \textit{cause-and-effect} description. In this example, 
we can describe state 0 \textit{leads to} state 3:\newline
\begin{tla}
    \* state 0 leads to state 3: TRUE 
    LeadsTo == counter = 0 ~> counter = 3

    \* state 0 leads to state 4: FALSE - model checker reports violation
    LeadsTo == counter = 0 ~> counter = 4
\end{tla}
\begin{tlatex}
\@x{\@s{16.4}}%
\@y{%
  state 0 leads to state 3: TRUE 
}%
\@xx{}%
 \@x{\@s{16.4} LeadsTo \.{\defeq} counter \.{=} 0 \.{\leadsto} counter \.{=}
 3}%
\@pvspace{8.0pt}%
\@x{\@s{16.4}}%
\@y{%
  state 0 leads to state 4: FALSE - model checker reports violation
}%
\@xx{}%
 \@x{\@s{16.4} LeadsTo \.{\defeq} counter \.{=} 0 \.{\leadsto} counter \.{=}
 4}%
\end{tlatex}
\newline

Note that \textit{leads to} is only evaluated if the left hand side is
\textit{true}. If right hand side is updated to counter = 4, the liveness
property will fail as expected. However, if left hand side is false, then the
liveness property is not evaluated since there isn't a state that satisfies the
cause condition. For example, the model checker will not report violation for
the following liveness property:\newline
\begin{tla}
    \* model checker will NOT report violation because cause condition never occur 
    LeadsTo == counter = 4 ~> counter = 3
\end{tla}
\begin{tlatex}
\@x{\@s{16.4}}%
\@y{%
  model checker will NOT report violation because cause condition never occur 
}%
\@xx{}%
 \@x{\@s{16.4} LeadsTo \.{\defeq} counter \.{=} 4 \.{\leadsto} counter \.{=}
 3}%
\end{tlatex}

% \end{document}
