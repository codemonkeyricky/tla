% \begin{document}

\chapter{Liveness}
\label{chap:liveness}

In this chapter we will go through a very simple state machine to demonstare
liveness properties. Assume a simple three state system:\newline

\begin{center}
\begin{tikzpicture}[>=stealth',shorten >=1pt,auto,node distance=2cm]
  \node[state]  (q1)                {1};
  \node[state]  (q2) [right of=q1]  {2};
  \node[state]  (q3) [right of=q2]  {3};

  \path[->]          (q1)  edge   []   node {} (q2);
%   \path[->]          (q2)  edge   [bend left=20]   node {} (q1);

  \path[->]          (q2)  edge   [bend left=20]   node {} (q3);
  \path[->]          (q3)  edge   [bend left=20]   node {} (q2);
\end{tikzpicture}
\end{center}

This can be described by the following Spec:\newline

\begin{tla}
--------------------------- MODULE liveness ----------------------------
EXTENDS Naturals
VARIABLES counter 
vars == <<counter>>

EventuallyAlways == <>[](counter = 3)
AlwaysEventually == []<>(counter = 3)

Init ==
    /\ counter = 0

Inc == 
    /\ counter' = counter + 1

Dec == 
    /\ counter' = counter - 1

Next ==
    \/ /\ counter # 3
       /\ Inc
    \/ /\ counter = 3
       /\ Dec

Spec ==
  /\ Init
  /\ [][Next]_vars
  /\ WF_vars(Next)
=============================================================================
\end{tla}
\begin{tlatex}
\@x{}\moduleLeftDash\@xx{ {\MODULE} liveness}\moduleRightDash\@xx{}%
\@x{ {\EXTENDS} Naturals}%
\@x{ {\VARIABLES} counter}%
\@x{ vars \.{\defeq} {\langle} counter {\rangle}}%
\@pvspace{8.0pt}%
\@x{ EventuallyAlways \.{\defeq} {\Diamond} {\Box} ( counter \.{=} 3 )}%
\@x{ AlwaysEventually \.{\defeq} {\Box} {\Diamond} ( counter \.{=} 3 )}%
\@pvspace{8.0pt}%
\@x{ Init \.{\defeq}}%
\@x{\@s{16.4} \.{\land} counter \.{=} 0}%
\@pvspace{8.0pt}%
\@x{ Inc \.{\defeq}}%
\@x{ \.{\land} counter \.{'} \.{=} counter \.{+} 1}%
\@pvspace{8.0pt}%
\@x{ Dec \.{\defeq}}%
\@x{ \.{\land} counter \.{'} \.{=} counter \.{-} 1}%
\@pvspace{8.0pt}%
\@x{ Next \.{\defeq}}%
\@x{\@s{16.4} \.{\lor} \.{\land} counter \.{\neq} 3}%
\@x{\@s{16.4} \.{\land} Inc}%
\@x{\@s{16.4} \.{\lor} \.{\land} counter \.{=} 3}%
\@x{\@s{16.4} \.{\land} Dec}%
\@pvspace{8.0pt}%
\@x{ Spec \.{\defeq}}%
\@x{\@s{8.2} \.{\land} Init}%
\@x{\@s{8.2} \.{\land} {\Box} [ Next ]_{ vars}}%
\@x{\@s{8.2} \.{\land} {\WF}_{ vars} ( Next )}%
\@x{}\bottombar\@xx{}%
\end{tlatex}

Note the required fairness description in the spec. Without fairness the spec is
allowed to stuttering and model checker cannot verify liveness properties. 

\section{Always Eventually}

We want to verify the system always eventually makes it to state 3. This can be
described by the following liveness property:\newline

\begin{tla}
    AlwaysEventually == []<>(counter = 3)
\end{tla}
\begin{tlatex}
 \@x{\@s{16.4} AlwaysEventually \.{\defeq} {\Box} {\Diamond} ( counter \.{=} 3
 )}%
\end{tlatex}
\newline

Once the system makes it to state 3, the system is stuck in a loop transitioning
state 2 and 3. It doesn't \textit{remain} in state 3, but it does \textit{always
eventually} make it to state 3. The system as described fulfills this liveness
property.

\section{Eventually Always}

However, the system does not \textit{eventually always} stay in state 3, because
the system toggles between state 2 and 3. This is described by the following
liveness property: 

\begin{tla}
    AlwaysEventually == []<>(counter = 3)
\end{tla}
\begin{tlatex}
 \@x{\@s{16.4} AlwaysEventually \.{\defeq} {\Box} {\Diamond} ( counter \.{=} 3
 )}%
\end{tlatex}

To satisfy this liveness property, we will need to remove the transition from 3
to 2: 
\begin{tla}
Next ==
    \/ /\ counter # 3
       /\ Inc
\end{tla}


\section{Leads To}

% \end{document}
