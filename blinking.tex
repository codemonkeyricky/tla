\chapter{Blinking LED}

Let's start with a trivial specification of a blinking LED. The intent of this example 
is to demonstrate the core functionalities of TLA+ specification language.

TODO: briefly talk about tla+ and model checker here.

\section{Requirement}

The LED is represented by a boolean variable that can be either 0 or 1.\newline

... that's it.

\section{Spec}

The specification language may appear alienating as it is mathematically
motivated based on propositional logic. Despite the (possibly) daunting syntax,
designer only need to be familiar with a handful of key operators to start
realizing value using TLA+. This chapter will attempt to describe the example in
exhaustive detail to reduce the learning curve.

The following describe the core portion of the blinking LED spec. 

\begin{tla}
--------------------------- MODULE blinking ----------------------------
VARIABLES b 
vars == <<b>>
Init ==
    /\ b = 0
On == 
    /\ b = 0
    /\ b' = 1
Off == 
    /\ b = 1
    /\ b' = 0
Next ==
    \/ Off 
    \/ On
Spec ==
    /\ Init
    /\ [][Next]_vars
========
\end{tla}
\begin{tlatex}
\@x{}\moduleLeftDash\@xx{ {\MODULE} blinking}\moduleRightDash\@xx{}%
\@x{ {\VARIABLES} b}%
\@x{ vars \.{\defeq} {\langle} b {\rangle}}%
\@x{ Init\@s{2.02} \.{\defeq}}%
\@x{\@s{16.4} \.{\land} b \.{=} 0}%
\@x{ On \.{\defeq}}%
\@x{\@s{18.15} \.{\land} b \.{=} 0}%
\@x{\@s{18.15} \.{\land} b \.{'} \.{=} 1}%
\@x{ Off \.{\defeq}}%
\@x{\@s{15.91} \.{\land} b \.{=} 1}%
\@x{\@s{15.91} \.{\land} b \.{'} \.{=} 0}%
\@x{ Next \.{\defeq}}%
\@x{\@s{16.4} \.{\lor} Off}%
\@x{\@s{16.4} \.{\lor} On}%
\@x{ Spec \.{\defeq}}%
\@x{\@s{16.4} \.{\land} Init}%
\@x{\@s{16.4} \.{\land} {\Box} [ Next ]_{ vars}}%
\@x{}\bottombar\@xx{}%
\end{tlatex}

\begin{itemize}
    \item $\defeq$ is the \textit{defines equal} operator 
    \item $\land$ and $\lor$ are the AND and OR operator. The effect
    of these operator follow the natural definition in English: 
    \begin{itemize}
        \item $C \defeq A \land B$: C is true iff A and B are true
        \item $C \defeq A \lor B$: C is true iff A or B is true
    \end{itemize}
    \item The $'$ operator represents the next state. $b'$ represent b's next state. 
    \item $VARIABLES$ keyword defines a list of variables for the spec. In this case 
    the spec defines a variable $b$ which can be either 0 or 1
    \item $vars$ is typically defined as a shorthand to refer to \textit{all}
    variables in the spec. 
\end{itemize}

With the above definition, we can revisit the Action definitions: $Init$ defines
the initial system state, where b is set to 0.\newline 

$Next$ requires more elaboration. TLA+ specifies the system as a collection of
states with transitions between them. In a simplified sense, the state is
described as a collection of ANDs (eg. system is in state C if both A and B are
true), the ORs then describe the states the system can possibly be in (eg.
system can be in state C OR D). Revisiting the example, the blinking LED has two
states:
\begin{itemize}
    \item $On \defeq b = 0 \land b' = 1$: b switches on 
    \item $Off \defeq = 1 \land b' = 0$: b switches off
\end{itemize}

The system's $Next$ state is defined to be one of these states:\newline
$Next \defeq On \lor Off$.\newline

$\Box[Next]_{vars}$ is a \textbf{Box-Action Formula}, where \textit{Next} is an
action and \textit{vars} is a state function. The formula is true iff every
successive pair of steps in behaviour is a $[Next]_{vars}$. Finally $Spec$ is
conjunction between $Init$ and $\Box[Next]_{vars}$. Note \textbf{all} TLA+
specification follows very similar template. There are situation we will need to
provide \textit{fairness} description - this will be covered later. \newline

In short: this specification describes a two-state state machine where b toggles
between 0 and 1.\newline

Note that b can technically be \textit{anything}. b can be 0, 1, -42, a
dinosaur, etc. TLA+ specifies values of $b$ which are valid in the system.

\section{Safety}

The spec so far only defines the possible states - but the \textit{power} of
TLA+ lies in its \textit{properties} description. Safety properties are
invariants that must hold true in \textit{every} state. An invariant in the
blinking LED example is: 
\begin{tla}
    TypeOK == b \in {0, 1}
\end{tla}
\begin{tlatex}
\@x{\@s{16.4} TypeOK \.{\defeq} b \.{\in} \{ 0 ,\, 1 \}}%
\end{tlatex}

This states the only valid value of b is 0 or 1. If b is ever set to anything
else, the spec is invalid.\newline

Some example safety properties include: Only a single thread have exclusive
access to critical section, number of concurrent reads cannot exceed data
available to be read, etc. 

\section{Liveness}

While safety properties describe invariant that must be upheld in every state,
\textit{Liveness} describe properties of a sequence of states. In the blinking
LED example, a liveness property can be the if b is 0, it eventually becomes 1,
and vice versa. This is described below:
\begin{tla}
    Liveness == 
        /\ b = 0 ~> b = 1
        /\ b = 1 ~> b = 0
\end{tla}
\begin{tlatex}
\@x{\@s{16.4} Liveness \.{\defeq}}%
\@x{\@s{32.8} \.{\land} b \.{=} 0 \.{\leadsto} b \.{=} 1}%
\@x{\@s{32.8} \.{\land} b \.{=} 1 \.{\leadsto} b \.{=} 0}%
\end{tlatex}

It is the author's opinion liveness describes the \textit{design essense} behind
the spec. The key characteristic of a system is described by its
\textit{behaviour} across a series of states. Does a distribute algorithm
eventually converge to a working state? Does a resource manager fairly allocate
resources in all scenarios? Does a scheduler ensure all tasks are eventually
scheduled? These are behaviours that are \textit{cannot} be concluded by looking
at a single state, but across a \textit{sequence of state}. Liveness allows 
designer to express and verify these properties.

\section{Model Checking}

Since the blinking LED is trivially specified, the full specification is
included below. For subsequent chapters only snippet will be included. Please
refer to the accompanied material for full spec source. 

TODO: install toolchain 

TODO: commandline

TODO: using TLC

The following is the content of \textit{blinking.tla}:
\begin{tla}
--------------------------- MODULE blinking ----------------------------
EXTENDS Naturals
VARIABLES b 
vars == <<b>>
TypeOK ==
  /\ b \in {0, 1} 
Liveness == 
    /\ b = 0 ~> b = 1
    /\ b = 1 ~> b = 0
Init ==
  /\ b = 0
Next ==
  \/ /\ b = 0
     /\ b' = 1
  \/ /\ b = 1
     /\ b' = 0
Spec ==
  /\ Init
  /\ [][Next]_vars
  /\ WF_vars(Next)
=============================================================================
\end{tla}
\begin{tlatex}
\@x{}\moduleLeftDash\@xx{ {\MODULE} blinking}\moduleRightDash\@xx{}%
\@x{ {\EXTENDS} Naturals}%
\@x{ {\VARIABLES} b}%
\@x{ vars \.{\defeq} {\langle} b {\rangle}}%
\@x{ TypeOK \.{\defeq}}%
\@x{\@s{8.2} \.{\land} b \.{\in} \{ 0 ,\, 1 \}}%
\@x{ Liveness \.{\defeq}}%
\@x{\@s{16.4} \.{\land} b \.{=} 0 \.{\leadsto} b \.{=} 1}%
\@x{\@s{16.4} \.{\land} b \.{=} 1 \.{\leadsto} b \.{=} 0}%
\@x{ Init \.{\defeq}}%
\@x{\@s{8.2} \.{\land} b \.{=} 0}%
\@x{ Next \.{\defeq}}%
\@x{\@s{8.2} \.{\lor}\@s{1.63} \.{\land} b \.{=} 0}%
\@x{\@s{20.94} \.{\land} b \.{'} \.{=} 1}%
\@x{\@s{8.2} \.{\lor}\@s{1.63} \.{\land} b \.{=} 1}%
\@x{\@s{20.94} \.{\land} b \.{'} \.{=} 0}%
\@x{ Spec \.{\defeq}}%
\@x{\@s{8.2} \.{\land}\@s{0.16} Init}%
\@x{\@s{8.2} \.{\land}\@s{0.16} {\Box} [ Next ]_{ vars}}%
\@x{\@s{8.2} \.{\land}\@s{0.16} {\WF}_{ vars} ( Next )}%
\@x{}\bottombar\@xx{}%
\end{tlatex}

The following is the content of \textit{blinking.cfg}:

\begin{lstlisting}
    SPECIFICATION Spec
    INVARIANTS TypeOK
    PROPERTIES Liveness
\end{lstlisting}

\section{Limitation}

Since TLA+ exhaustively explores all possible state, a linear growth of
variables leads to TLC (temporal logic checker) execution time grows
\textit{exponentially}.This means the specification must be scoped correctly to
limit the state space.\newline

Similarly, if you want to verify concurrent psuedo code implementation in
PlusCal, you can likely at most verify 10s of lines of code.

