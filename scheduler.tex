% \begin{document}

\chapter{Simple Scheduler}

\section{Design}

In this section we will define a spec for a simple task scheduler. The task sechdeuler has the following
requirements:
\begin{itemize}
    \item Supporting N execution context (ie. CPUs)
    \item Supporting T number of tasks
    \item Tasks have identical priority and are scheduled coopertively
    \item System has a single global lock
    \item Any task can attempt to acquire the lock, Any task attempting to
    acquire the lock are gauranteed to be scheduled.
    \item If multiple tasks attempt to grab the lock, the tasks will be
    scheduled in lock request order. 
\end{itemize}

\section{Spec}

We will model scheduler using the following variables:\newline
\begin{tla}
Init ==
    /\ cpus = [i \in 0..N-1 |-> ""] 
    /\ ready_q = S2T(Tasks)
    /\ blocked_q = <<>>
    /\ lock_owner = ""
\end{tla}
\begin{tlatex}
\@x{ Init \.{\defeq}}%
 \@x{\@s{16.4} \.{\land} cpus \.{=} [ i \.{\in} 0 \.{\dotdot} N \.{-} 1
 \.{\mapsto}\@w{} ]}%
\@x{\@s{16.4} \.{\land} ready\_q \.{=} S2T ( Tasks )}%
\@x{\@s{16.4} \.{\land} blocked\_q \.{=} {\langle} {\rangle}}%
\@x{\@s{16.4} \.{\land} lock\_owner \.{=}\@w{}}%
\end{tlatex}
\newline

A few things to note:
\begin{itemize}
    \item The system has $N$ executing context, represented as number of CPUs.
    When a task is running, $cpus[k]$ is set to $taskName$. When CPU is idle,
    $cpus[k]$ is set to an empty string. 
    \item $ready\_q$ and $blocked\_q$ are initialized as \textit{ordered tuple},
    due to the cooperative scheduling requirement.
    \item $S2T$ is a macro that converts a set into a ordered tuple. This is to
    accommodate the fact it appears I cannot define tuple in .cfg file.
    \item Finally, the single system lock is represented as $lock\_owner$. 
\end{itemize}

A task can be in three possible state: Ready, Blocked and Running. The $Next$
box-action fomula will define a Ready and Running action, and the implementation
will include related lock contention handling.\newline

\begin{tla}
------- MODULE scheduler ---- 
MoveToReady(k) == 
    /\ cpus[k] # "" 
    /\ lock_owner # cpus[k]
    /\ ready_q' = Append(ready_q, cpus[k]) 
    /\ cpus' = [cpus EXCEPT ![k] = ""]
    /\ UNCHANGED <<lock_owner, blocked_q>>
Lock(k) == 
    \* lock is empty
    \/  /\ cpus[k] # "" 
        /\ lock_owner = ""
        /\ lock_owner' = cpus[k]
        /\ UNCHANGED <<ready_q, cpus, blocked_q>>
    \* someone else has the lock
    \/  /\ cpus[k] # "" 
        /\ lock_owner # ""
        /\ lock_owner # cpus[k] \* cannot double lock
        /\ blocked_q' = Append(blocked_q, cpus[k])
        /\ cpus' = [cpus EXCEPT ![k] = ""]
        /\ UNCHANGED <<ready_q, lock_owner>>
Unlock(k) == 
    /\ cpus[k] # "" 
    /\ lock_owner = cpus[k]
    /\ lock_owner' = ""
    /\ cpus' = [cpus EXCEPT ![k] = ""]
    /\ ready_q' = ready_q \o blocked_q \o <<cpus[k]>>
    /\ blocked_q' = <<>>

Running == 
    \E k \in DOMAIN cpus:
        /\ cpus[k] # "" 
        /\ \/ MoveToReady(k)
           \/ Lock(k)
           \/ Unlock(k)
=============================================================================
\end{tla}
\begin{tlatex}
\@x{}\moduleLeftDash\@xx{ {\MODULE} scheduler}\moduleRightDash\@xx{}%
\@x{ MoveToReady ( k ) \.{\defeq}}%
\@x{\@s{16.4} \.{\land} cpus [ k ] \.{\neq}\@w{}}%
\@x{\@s{16.4} \.{\land} lock\_owner \.{\neq} cpus [ k ]}%
 \@x{\@s{16.4} \.{\land} ready\_q \.{'} \.{=} Append ( ready\_q ,\, cpus [ k ]
 )}%
 \@x{\@s{16.4} \.{\land} cpus \.{'} \.{=} [ cpus {\EXCEPT} {\bang} [ k ]
 \.{=}\@w{} ]}%
 \@x{\@s{16.4} \.{\land} {\UNCHANGED} {\langle} lock\_owner ,\, blocked\_q
 {\rangle}}%
\@x{ Lock ( k ) \.{\defeq}}%
\@x{}%
\@y{%
  lock is empty
}%
\@xx{}%
\@x{ \.{\lor}\@s{4.1} \.{\land} cpus [ k ] \.{\neq}\@w{}}%
\@x{\@s{4.1} \.{\land} lock\_owner \.{=}\@w{}}%
\@x{\@s{4.1} \.{\land} lock\_owner \.{'} \.{=} cpus [ k ]}%
 \@x{\@s{4.1} \.{\land} {\UNCHANGED} {\langle} ready\_q ,\, cpus ,\,
 blocked\_q {\rangle}}%
\@x{}%
\@y{%
  someone else has the lock
}%
\@xx{}%
\@x{ \.{\lor}\@s{4.1} \.{\land} cpus [ k ] \.{\neq}\@w{}}%
\@x{\@s{4.1} \.{\land} lock\_owner \.{\neq}\@w{}}%
\@x{\@s{4.1} \.{\land} lock\_owner \.{\neq} cpus [ k ]}%
\@y{%
  cannot double lock
}%
\@xx{}%
 \@x{\@s{4.1} \.{\land} blocked\_q \.{'} \.{=} Append ( blocked\_q ,\, cpus [
 k ] )}%
 \@x{\@s{4.1} \.{\land} cpus \.{'} \.{=} [ cpus {\EXCEPT} {\bang} [ k ]
 \.{=}\@w{} ]}%
 \@x{\@s{4.1} \.{\land} {\UNCHANGED} {\langle} ready\_q ,\, lock\_owner
 {\rangle}}%
\@x{ Unlock ( k ) \.{\defeq}}%
\@x{\@s{16.4} \.{\land} cpus [ k ] \.{\neq}\@w{}}%
\@x{\@s{16.4} \.{\land} lock\_owner \.{=} cpus [ k ]}%
\@x{\@s{16.4} \.{\land} lock\_owner \.{'} \.{=}\@w{}}%
 \@x{\@s{16.4} \.{\land} cpus \.{'} \.{=} [ cpus {\EXCEPT} {\bang} [ k ]
 \.{=}\@w{} ]}%
 \@x{\@s{16.4} \.{\land} ready\_q \.{'} \.{=} ready\_q \.{\circ} blocked\_q
 \.{\circ} {\langle} cpus [ k ] {\rangle}}%
\@x{\@s{16.4} \.{\land} blocked\_q \.{'} \.{=} {\langle} {\rangle}}%
\@pvspace{8.0pt}%
\@x{ Running \.{\defeq}}%
\@x{\@s{16.4} \E\, k \.{\in} {\DOMAIN} cpus \.{:}}%
\@x{\@s{20.5} \.{\land} cpus [ k ] \.{\neq}\@w{}}%
\@x{\@s{20.5} \.{\land} \.{\lor} MoveToReady ( k )}%
\@x{\@s{20.5} \.{\lor} Lock ( k )}%
\@x{\@s{20.5} \.{\lor} Unlock ( k )}%
\@x{}\bottombar\@xx{}%
\end{tlatex}

\section{Safety}

\section{Liveness}

I believe this is the most important part of cooperative scheduler design. While
the scheduler can't \textit{force} a task to relinquish a lock (the scheduler
doesn't dictate when the task is \textit{done}), the scheduler can ensure
scheduling fairness by scheduling the next lock requester intsead of the task 
that just relinquished the lock.\newline

\begin{tla}
----- MODULE scheduler ---- 
Liveness == 
    \A t \in Tasks:
        LET 
            b == {x \in DOMAIN blocked_q : blocked_q[x] = t}
        IN 
            /\ b # {} ~> b = {}
==========
\end{tla}
\begin{tlatex}
\@x{}\moduleLeftDash\@xx{ {\MODULE} scheduler}\moduleRightDash\@xx{}%
\@x{ Liveness \.{\defeq}}%
\@x{\@s{16.4} \A\, t \.{\in} Tasks \.{:}}%
\@x{\@s{20.5} \.{\LET}}%
 \@x{\@s{36.9} b \.{\defeq} \{ x \.{\in} {\DOMAIN} blocked\_q \.{:} blocked\_q
 [ x ] \.{=} t \}}%
\@x{\@s{20.5} \.{\IN}}%
\@x{\@s{36.9} \.{\land} b \.{\neq} \{ \} \.{\leadsto} b \.{=} \{ \}}%
\@x{}\bottombar\@xx{}%
\end{tlatex}
\newline

The formula defines set b to be either an empty set or a set of one task. Assume
a set of \{"p0", "p1", "p2\}. Possible value of b include: \{\}, \{"p0"\},
\{"p1"\} and \{"p2"\}. The fomula then states an non empty set of b leads to an
\textit{empty set} of b. In other words: \newline

\textit{If a task ever becomes blocked, it will eventually become unblocked}.\newline

However, when we actually run the model checker, we will find the liveness
property is \textit{violated}. The failure scenario is basically one task
holding onto the lock in one CPU, while the scheduler repeatedly
schedule/deschedule a separate task in another CPU. While this is perfectly
allowed, the model checker detects a possible path for the the system to trap in
a local state and fail the liveness property.\newline 

Perhaps not surprisingly, if you construct similar liveness property to verify 
a task is \textit{eventually} always scheduled, it will also fail. The model
checker will provide a counter case where a task is never scheduled because
another task is repeatedly acquire/release the global lock.\newline

We need \textit{Strong Fairness} to solve this problem:
\begin{tla}
----- MODULE scheduler ---- 
L ==
    \A t \in Tasks :
        \A n \in 0..(N-1) :
            WF_vars(HoldingLock(t) /\ Unlock(n))
Spec ==
  /\ Init
  /\ [][Next]_vars
  /\ WF_vars(Next)
  /\ L 
======= 
\end{tla}
\begin{tlatex}
\@x{}\moduleLeftDash\@xx{ {\MODULE} scheduler}\moduleRightDash\@xx{}%
\@x{ L \.{\defeq}}%
\@x{\@s{8.2} \A\, t \.{\in} Tasks \.{:}}%
\@x{\@s{12.29} \A\, n \.{\in} 0 \.{\dotdot} ( N \.{-} 1 ) \.{:}}%
\@x{\@s{16.4} {\WF}_{ vars} ( HoldingLock ( t ) \.{\land} Unlock ( n ) )}%
\@x{ Spec \.{\defeq}}%
\@x{\@s{8.2} \.{\land} Init}%
\@x{\@s{8.2} \.{\land} {\Box} [ Next ]_{ vars}}%
\@x{\@s{8.2} \.{\land} {\WF}_{ vars} ( Next )}%
\@x{\@s{8.2} \.{\land} L}%
\@x{}\bottombar\@xx{}%
\end{tlatex}
\newline

Fairness ensures that we are never stuck in a repeated state.

% \end{document}
