% \begin{document}

\chapter{SPMC Lockless Queue}

As the name suggest, a SPMC lockless queue supports a single producer multiple
consumer usage topology.\newline

In previously described SPSC, the reader assumes write index is reserved, and
the writer assumes read index is reserved. While the relationship between read
and write context remains true in SPMC, the complication is now readers compete
to reserve an index during read. Some design considerations include:
\begin{itemize}
    \item Readers must have a way to reserve an index to read
    \item Readers may complete the read in order different from when they reserve the indices 
    \item Readers can perform the reads independently
\end{itemize}

One of the primitives in modern CPU architecture is compare-and-swap (CAS). In
short, to issue a CAS the issuer needs to specify a memory address, an old value
and a new value. Only when value in the memory location matches the old value,
the new value will be writen. The instruction ensure this is done atomically.
Assume two CPUs issuing CAS at the same time, only one CPU will \textit{win} as 
the other CPU's old value check will fail. This will be a key part of our SPMC design.

\section{Design}

Our solution space is bound by a few requirements driven by the primitives provided by modern CPU:
\begin{itemize}
    \item A resource is exclusively updated by one owner, and read by one or more readers
    \item For shared resource that can be updated by multiple owners, the CPU
    can gaurantees exclusive update from a single resource 
\end{itemize}

For the design:
\begin{itemize}
    \item SPMC is implemented as a circular queue with size of N
    \item The status of indvidual index is represented as an status array of size N
    \item The status of each index is either UNUSED, WRITTEN, or READING
    \item Each reader maintain its own read pointer
    \item A \textit{outstanding} counter is incremented by the writer when write is complete, 
    and decremented by the reader when it reserves a read
\end{itemize}

Whenever the write finishes a write, it updates the ready to indicate some
buffer is ready to read. To read, a reader performs a two-step reservation: 
\begin{itemize}
    \item The reader first decrement \textit{outstanding}. A successful decrement means
    the reader is \textit{gauranteed} a read index
    \item After successful decrement of \textit{outstanding}, the reader walk
    its read pointer until it successfully reserve the next available index to
    read. A This is done by issuing CAS to update an index from WRITTEN to READING
\end{itemize}

One of the complication is writer full check. In SPSC, when write pointer
catches up to read pointer, the queue is full. In SPMC, there's multiple
readers, making this check difficult to implement without a separate atomic
variable.

\section{Spec}

\begin{pcal}
procedure reader() 
variable 
    i = self;
begin
r_chk_empty:        
    if outstanding # 0 then 
        outstanding := outstanding - 1; 
    else 
    r_early_ret:            
        return;
    end if;
\* reserved a read - now find it
r_try_lock:        
        status[rptr[i]] := READING;
    else 
    r_retry:                
        rptr[i] := (rptr[i] + 1) % N;
            goto r_try_lock;
    end if;
r_data_chk:         
    assert buffer[rptr[i]] = rptr[i] + 1000;
r_read_buf:         
    buffer[rptr[i]] := 0;
r_unlock:           
    status[rptr[i]] := UNUSED;
r_done:             
    return;
end procedure; 

procedure writer() begin
w_chk_full:         
    if outstanding = N - 1 then 
    w_early_ret:            
        return; 
    end if;
w_chk_st:           
    if status[wptr] # UNUSED then 
    w_early_ret2:           
        return;
    end if;
w_write_buf:        
    buffer[wptr] := wptr + 1000;
w_mark_written:     
    status[wptr] := WRITTEN;
w_inc_wptr:         
    wptr := (wptr + 1) % N;
w_inc:              
    outstanding := outstanding + 1;
w_done:             
    return;
end procedure; 
\end{pcal}
\begin{tlatex}
\@x{ {\p@procedure} reader {\p@lparen} {\p@rparen}}%
\@x{ {\p@variable}}%
\@x{\@s{16.4} i \.{=} self {\p@semicolon}}%
\@x{ {\p@begin}}%
\@x{ r\_chk\_empty\@s{.5}\textrm{:}\@s{3}}%
\@x{\@s{16.4} {\p@if} outstanding \.{\neq} 0 {\p@then}}%
\@x{\@s{20.5} outstanding \.{:=} outstanding \.{-} 1 {\p@semicolon}}%
\@x{\@s{16.4} {\p@else}}%
\@x{\@s{16.4} r\_early\_ret\@s{.5}\textrm{:}\@s{3}}%
\@x{\@s{32.8} {\p@return} {\p@semicolon}}%
\@x{\@s{16.4} {\p@end} {\p@if} {\p@semicolon}}%
\@x{}%
\@y{%
  reserved a read - now find it
}%
\@xx{}%
\@x{ r\_try\_lock\@s{.5}\textrm{:}\@s{3}}%
\@x{\@s{16.4} {\p@if} status [ rptr [ i ] ] \.{=} WRITTEN {\p@then}}%
\@x{\@s{20.5} status [ rptr [ i ] ] \.{:=} READING {\p@semicolon}}%
\@x{\@s{16.4} {\p@else}}%
\@x{\@s{16.4} r\_retry\@s{.5}\textrm{:}\@s{3}}%
 \@x{\@s{32.8} rptr [ i ] \.{:=} ( rptr [ i ] \.{+} 1 ) \.{\%} N
 {\p@semicolon}}%
\@x{\@s{32.8} {\p@goto} r\_try\_lock {\p@semicolon}}%
\@x{\@s{16.4} {\p@end} {\p@if} {\p@semicolon}}%
\@x{ r\_data\_chk\@s{.5}\textrm{:}\@s{3}}%
 \@x{\@s{16.4} {\p@assert} buffer [ rptr [ i ] ] \.{=} rptr [ i ] \.{+} 1000
 {\p@semicolon}}%
\@x{ r\_read\_buf\@s{.5}\textrm{:}\@s{3}}%
\@x{\@s{16.4} buffer [ rptr [ i ] ] \.{:=} 0 {\p@semicolon}}%
\@x{ r\_unlock\@s{.5}\textrm{:}\@s{3}}%
\@x{\@s{16.4} status [ rptr [ i ] ] \.{:=} UNUSED {\p@semicolon}}%
\@x{ r\_done\@s{.5}\textrm{:}\@s{3}}%
\@x{\@s{16.4} {\p@return} {\p@semicolon}}%
\@x{ {\p@end} {\p@procedure} {\p@semicolon}}%
\@pvspace{8.0pt}%
\@x{ {\p@procedure} writer {\p@lparen} {\p@rparen} {\p@begin}}%
\@x{ w\_chk\_full\@s{.5}\textrm{:}\@s{3}}%
\@x{\@s{16.4} {\p@if} outstanding \.{=} N \.{-} 1 {\p@then}}%
\@x{\@s{16.4} w\_early\_ret\@s{.5}\textrm{:}\@s{3}}%
\@x{\@s{32.8} {\p@return} {\p@semicolon}}%
\@x{\@s{16.4} {\p@end} {\p@if} {\p@semicolon}}%
\@x{ w\_chk\_st\@s{.5}\textrm{:}\@s{3}}%
\@x{\@s{16.4} {\p@if} status [ wptr ] \.{\neq} UNUSED {\p@then}}%
\@x{\@s{16.4} w\_early\_ret2\@s{.5}\textrm{:}\@s{3}}%
\@x{\@s{32.8} {\p@return} {\p@semicolon}}%
\@x{\@s{16.4} {\p@end} {\p@if} {\p@semicolon}}%
\@x{ w\_write\_buf\@s{.5}\textrm{:}\@s{3}}%
\@x{\@s{16.4} buffer [ wptr ] \.{:=} wptr \.{+} 1000 {\p@semicolon}}%
\@x{ w\_mark\_written\@s{.5}\textrm{:}\@s{3}}%
\@x{\@s{16.4} status [ wptr ] \.{:=} WRITTEN {\p@semicolon}}%
\@x{ w\_inc\_wptr\@s{.5}\textrm{:}\@s{3}}%
\@x{\@s{16.4} wptr \.{:=} ( wptr \.{+} 1 ) \.{\%} N {\p@semicolon}}%
\@x{ w\_inc\@s{.5}\textrm{:}\@s{3}}%
\@x{\@s{16.4} outstanding \.{:=} outstanding \.{+} 1 {\p@semicolon}}%
\@x{ w\_done\@s{.5}\textrm{:}\@s{3}}%
\@x{\@s{16.4} {\p@return} {\p@semicolon}}%
\@x{ {\p@end} {\p@procedure} {\p@semicolon}}%
\end{tlatex}

% \end{document}
