\chapter{Simple Gossip Protocol}

This section the author's notes on a simple gossip protocol by Andrew Hewler:\newline
https://ahelwer.ca/post/2023-11-01-tla-finite-monotonic/\newline

\section{Design}

In a distributed system, a cluster of nodes collectively provide a serivce. A
distributed database may have a collection of 10s to 100s of nodes working
together to offer the service in a geo diverse fashion to be immnue to partial
outage.  The nodes often have requirements to know about each other. In the
context of distributed database, a node may need to know the key range another
of its peers. The cluster needs a way to communicate this information. One such
mechanism is the gossip protocol.\newline

Gossip protocols are used to communicate cluster information in a distributed
fashion, (unsurprisingly) in a distributed system. Without gossip protocol, 
nodes in a cluster learns about its neighbours by contacting a centralized
server. This introduces a single failure point in the system. As the name
suggest, gossip protocol relies on nodes to gossip with each other. The nodes in
the cluster periodically selects a set of neighbors to exchange what it knows
about the cluster. The recency information is part of the gossip message
itself, allowing the node and the peer its talking to quickly decide who has the
latest information on a node, and converge to it. Assume a N node cluster and
each interal a node selects k neighbours to gossip with, the total amount of
gossip propagation time is described logrithmticly below:

\begin{equation} 
    propagation\_time = \log_k N * gossip\_interval
\end{equation}

With the total number of messages exchanged: 
\begin{equation} 
    messages\_exchanged = \log_k N * k
\end{equation}

Now let's look at how a simple gossip protocol can be described by TLA+.

\section{Spec}

In gossip protocol, every node needs to remember all its peers current
version. This can be represented as a two dimension array:\newline
\begin{tla}
    Init == counter = [n \in Node |-> [o \in Node |-> 0]] 
\end{tla}
\begin{tlatex}
 \@x{\@s{16.4} Init \.{\defeq} counter \.{=} [ n \.{\in} Node \.{\mapsto} [ o
 \.{\in} Node \.{\mapsto} 0 ] ]}%
\end{tlatex}
\newline

This defines counter a collection of nodes, where each node also contains a
collection of nodes initialized to 0. The nodes can bump to a new version:\newline
\begin{tla}
    Increment(n) == counter' = [counter EXCEPT ![n][n] = @ + 1]
\end{tla}
\begin{tlatex}
 \@x{\@s{16.4} Increment ( n ) \.{\defeq} counter \.{'} \.{=} [ counter
 {\EXCEPT} {\bang} [ n ] [ n ] \.{=} @ \.{+} 1 ]}%
\end{tlatex}
\newline

Notice increment only bumps node's version. This change needs to be gossiped 
across the cluster:\newline
\begin{tla}
Gossip(n, o) ==                  
    LET Max(a, b) == IF a > b THEN a ELSE b 
    IN counter' = [
        counter EXCEPT ![o] = [
            nn \in Node |->            
                Max(counter[n][nn], counter[o][nn])
            ] 
    ]
\end{tla}
\begin{tlatex}
\@x{ Gossip ( n ,\, o ) \.{\defeq}}%
 \@x{\@s{16.4} \.{\LET} Max ( a ,\, b ) \.{\defeq} {\IF} a \.{>} b \.{\THEN} a
 \.{\ELSE} b}%
\@x{\@s{16.4} \.{\IN} counter \.{'} \.{=} [}%
\@x{\@s{40.89} counter {\EXCEPT} {\bang} [ o ] \.{=} [}%
\@x{\@s{57.29} nn \.{\in} Node \.{\mapsto}}%
\@x{\@s{73.41} Max ( counter [ n ] [ nn ] ,\, counter [ o ] [ nn ] )}%
\@x{\@s{57.29} ]}%
\@x{\@s{16.4} ]}%
\end{tlatex}
\newline

A few things to unpack here:
\begin{itemize}
    \item $n, o$ are the two nodes exchanging gossip. $o$ is the node to be updated
    and $n$ is the neighbor $o$ gossips with.
    \item $LET .. IN$ allows local definition under $LET$ used under
    $IN$. In this case $Max$ is a local macro defined to return maximum between a and b.
    \item $counter'$ (or refered to as counter \textit{prime}) is what the
    variable will be in the next state. TLA+ doesn't provide a way to update a
    variable in a collection, so the convention is to assign a new array to the variable. 
    \item $counter\ EXCEPT ![o] = [...]$ return $counter$ with $counter[o]$
    defined in the bracket. 
    \item where $[...]$ is a collection of nodes with with counter set to the
    max between the current node and neighbour.
\end{itemize}

Finally, the actual spec: \newline
\begin{tla}
    Next == \/ \E n \in Node : Increment(n)
            \/ \E n, o \in Node : Gossip(n, o)
\end{tla}
\begin{tlatex}
 \@x{\@s{16.4} Next \.{\defeq} \.{\lor} \E\, n \.{\in} Node \.{:} Increment (
 n )}%
\@x{\@s{56.23} \.{\lor} \E\, n ,\, o \.{\in} Node \.{:} Gossip ( n ,\, o )}%
\end{tlatex}
\newline

Next supports two possible next steps describe using disjunctions. The first is
bumping the version of a random node, the second is select a pair of nodes to
gossip. Note the \textit{existential qualifier} on both, which basically states
there exists a node n in nodes, or there exists a pair of nodes n, o in nodes,
respectively.\newline

There's a minor problem with the definition above. Gossip protocol, like many
converging protocols, have a \textit{monotonic increasing} requirement. On
failures, the protocol bumps the version, which increases monotonically. Since
TLA+ spec models the system as a graph, a monotonic increasing version number
means the state graph is \textit{infinitely large}. To put the specification back into
finite space, we can normalize the state:\newline

\begin{tla}
GarbageCollect ==
    LET SetMin(s) == CHOOSE e \in s : \A o \in s : e <= o IN
    LET Transpose == SetMin({counter[n][o] : n, o \in Node}) IN
        /\ counter' = [
            n \in Node |-> [
                o \in Node |-> counter[n][o] - Transpose
            ]
          ]
        /\ UNCHANGED converge
\end{tla}
\begin{tlatex}
\@x{ GarbageCollect \.{\defeq}}%
 \@x{\@s{16.4} \.{\LET} SetMin ( s ) \.{\defeq} {\CHOOSE} e \.{\in} s \.{:}
 \A\, o \.{\in} s \.{:} e \.{\leq} o \.{\IN}}%
 \@x{\@s{16.4} \.{\LET} Transpose\@s{0.58} \.{\defeq} SetMin ( \{ counter [ n
 ] [ o ] \.{:} n ,\, o \.{\in} Node \} ) \.{\IN}}%
\@x{\@s{36.79} \.{\land} counter \.{'} \.{=} [}%
\@x{\@s{52.01} n \.{\in} Node \.{\mapsto} [}%
 \@x{\@s{66.60} o \.{\in} Node \.{\mapsto} counter [ n ] [ o ] \.{-}
 Transpose}%
\@x{\@s{52.01} ]}%
\@x{\@s{44.99} ]}%
\@x{\@s{36.79} \.{\land} {\UNCHANGED} converge}%
\end{tlatex}
\newline

\textit{GarbageCollect} substracts every version value with set minimum. To
limit range of version value, the increment function is now updated to: \newline
\begin{tla}
Increment(n) ==
    /\ ~converge
    /\ counter[n][n] < Divergence
    /\ S!Increment(n)
    /\ UNCHANGED converge
\end{tla}
\begin{tlatex}
\@x{ Increment ( n ) \.{\defeq}}%
\@x{\@s{16.4} \.{\land} {\lnot} converge}%
\@x{\@s{16.4} \.{\land} counter [ n ] [ n ] \.{<} Divergence}%
\@x{\@s{16.4} \.{\land} S {\bang} Increment ( n )}%
\@x{\@s{16.4} \.{\land} {\UNCHANGED} converge}%
\end{tlatex}
\newline

Finally, the \textit{Next} is updated to the follow:\newline

\begin{tla}
Next ==
    \/ \E n \in Node : Increment(n)
    \/ \E n, o \in Node : Gossip(n, o)
    \/ Converge
    \/ GarbageCollect
\end{tla}
\begin{tlatex}
\@x{ Next \.{\defeq}}%
\@x{\@s{16.4} \.{\lor} \E\, n \.{\in} Node \.{:} Increment ( n )}%
\@x{\@s{16.4} \.{\lor} \E\, n ,\, o \.{\in} Node \.{:} Gossip ( n ,\, o )}%
\@x{\@s{16.4} \.{\lor} Converge}%
\@x{\@s{16.4} \.{\lor} GarbageCollect}%
\end{tlatex}

Note $GarbageCollect$ is a now part of possible state transition. We will
discuss $Converge$ later, as it is related to liveness check. Lastly:\newline

\begin{tla}
Fairness == \A n, o \in Node : WF_vars(Gossip(n, o))
Spec ==
    /\ Init
    /\ [][Next]_vars
    /\ Fairness
\end{tla}
\begin{tlatex}
 \@x{ Fairness \.{\defeq} \A\, n ,\, o \.{\in} Node \.{:} {\WF}_{ vars} (
 Gossip ( n ,\, o ) )}%
\@x{ Spec \.{\defeq}}%
\@x{\@s{16.4} \.{\land} Init}%
\@x{\@s{16.4} \.{\land} {\Box} [ Next ]_{ vars}}%
\@x{\@s{16.4} \.{\land} Fairness}%
\end{tlatex}
\newline

The \textit{Fairness} definition ensures Gossip runs between every pair of nodes
gossip.

\section{Safety}

In every state, counter[n][o] next must be larger than counter[n][o] current:\newline

\begin{tla}
Monotonic == \A n, o \in Node : counter'[n][o] >= counter[n][o] 
Monotonicity == [][
    \/ S!Monotonic
    \/ \A a, b, c, d \in Node :
        (counter'[a][b] - counter[a][b]) = (counter'[c][d] - counter[c][d])
]_vars
\end{tla}
\begin{tlatex}
 \@x{ Monotonic \.{\defeq} \A\, n ,\, o \.{\in} Node \.{:} counter \.{'} [ n ]
 [ o ] \.{\geq} counter [ n ] [ o ]}%
\@x{ Monotonicity \.{\defeq} {\Box} [}%
\@x{\@s{16.4} \.{\lor} S {\bang} Monotonic}%
\@x{\@s{16.4} \.{\lor} \A\, a ,\, b ,\, c ,\, d \.{\in} Node \.{:}}%
 \@x{\@s{31.61} ( counter \.{'} [ a ] [ b ] \.{-} counter [ a ] [ b ] ) \.{=}
 ( counter \.{'} [ c ] [ d ] \.{-} counter [ c ] [ d ] )}%
\@x{ ]_{ vars}}%
\end{tlatex}

\section{Liveness}

For liveness we want to check the version value across all nodes eventually
converge. \textit{Next} is updated to set Converge to true, which triggers the
liveness condition and ensure all pair of nodes eventually have the same
information. \newline

\begin{tla}
Convergence == \A n, o \in Node : counter[n] = counter[o]
Liveness == converge ~> S!Convergence
Converge ==
    /\ converge' = TRUE
    /\ UNCHANGED counter  
Next ==
  \/ \E n \in Node : Increment(n)
  \/ \E n, o \in Node : Gossip(n, o)
  \/ Converge
  \/ GarbageCollect
\end{tla}
\begin{tlatex}
 \@x{ Convergence \.{\defeq} \A\, n ,\, o \.{\in} Node \.{:} counter [ n ]
 \.{=} counter [ o ]}%
 \@x{ Liveness\@s{2.98} \.{\defeq} converge \.{\leadsto} S {\bang}
 Convergence}%
\@x{ Converge \.{\defeq}}%
\@x{\@s{16.4} \.{\land} converge \.{'} \.{=} {\TRUE}}%
\@x{\@s{16.4} \.{\land} {\UNCHANGED} counter}%
\@x{ Next \.{\defeq}}%
\@x{\@s{8.2} \.{\lor}\@s{1.63} \E\, n \.{\in} Node \.{:} Increment ( n )}%
 \@x{\@s{8.2} \.{\lor}\@s{1.63} \E\, n ,\, o \.{\in} Node \.{:} Gossip ( n ,\,
 o )}%
\@x{\@s{8.2} \.{\lor}\@s{1.63} Converge}%
\@x{\@s{8.2} \.{\lor}\@s{1.63} GarbageCollect}%
\end{tlatex}

