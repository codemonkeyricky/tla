\chapter{TLA+ Primer}

\section{Purpose}

The key insight to TLA+ is modeling a system as a state machine. A simple
digital clock can be represented by two variables: hour and minute. The
number of possible states in a digital clock is $24 * 60 = 1440$. For example, a
clock in state 10:00 will transition to state 10:01. Assume an arbitrary system
described by N variables, each variable having K possible values. Such an
arbitrary system can have up to $N^K$ states.\\

For every specification, the designer can specify \textit{safety} properties (or
invariants) that must be true in \textit{every} state. For example, in any
state of the digital clock, hour \textit{must} be between 0 to 23, or formally
described as $hour \in 0..23$. Similarly, the minute must have a value between 0 to
59, or $minute \in 0..59$. Examples of invariants of a system include: Only one
thread has exclusive access to a critical region, all variables in the system
are within allowable value, and the resource allocation manager never allocates more
than available resources.\\

The designer can also specify \textit{liveness} properties. These are properties to be
satisfied by a \textit{sequence of states}. One liveness property of the digital
clock could be when the clock is 10:00, it will eventually become 11:00 (10:00
\textit{leads} to 11:00). Example liveness property include: a distributed
system eventually converges, the scheduler eventually schedules every task in
the task queue, and the resource allocation manager fairly allocates resources.
\\

A TLA+ Spec can be checked by TLC, the model checker. TLC uses
\textit{breadth-first search} algorithm to explore \textit{all} states in the
state machine and ensure safety and liveness properties are upheld.\\

A TLA+ Spec describes the system using \textit{temporal logic}. The syntax may 
appear unfamiliar, but like any other programming 
language an un-initiated reader should become familiarized quickly.

\section{Design}

In this example, we will specify a \textit{digital clock}. The digital clock has
a few simple requirements:
\begin{itemize}
\item Two variables to represent state: hour and minute
\item The clock increments one minute at a time
\item Hour is between 0 to 23, inclusive
\item Minute is between 0 to 59, inclusive
\item Clock wraps around at midnight (ie. 23:59 transitions to 00:00)
\end{itemize}

\section{Spec}

The \textit{Init} state of such a system can be described as: \newline
\begin{tla}
    vars == <<hour, minute>>
    Init ==
        /\ hour = 0
        /\ minute = 0
\end{tla}
\begin{tlatex}
\@x{\@s{16.4} Init \.{\defeq}}%
\@x{\@s{32.8} \.{\land} hour \.{=} 0}%
\@x{\@s{32.8} \.{\land} minute \.{=} 0}%
\end{tlatex}
 \newline

$\defeq$ is the \textit{defines equal} symbol and $\land$ is the \textit{logical
and} symbol. The above TLA+ syntax can be read as \textit{Init} state is defined
as both hour and minute are 0.\newline

The spec also always includes a \textit{Next} definition, an \textit{action
formula} describing how the system transitions from one state to another. Action
formula contains \textit{primed} variables, representing values of variables in
their next state. The \textit{Next} action for the digital clock can be defined
as:\\

\begin{tla}
    NextHour ==
        /\ minute = 59 
        /\ hour' = (hour + 1) % 24
        /\ minute' = 0
    NextMinute == 
        /\ minute # 59
        /\ hour' = hour 
        /\ minute' = minute + 1 
    Next ==
        \/ NextMinute
        \/ NextHour
\end{tla}
\begin{tlatex}
\@x{\@s{16.4} NextHour \.{\defeq}}%
\@x{\@s{32.8} \.{\land} minute \.{=} 59}%
\@x{\@s{32.8} \.{\land} hour \.{'} \.{=} ( hour \.{+} 1 ) \.{\%} 24}%
\@x{\@s{32.8} \.{\land} minute \.{'} \.{=} 0}%
\@x{\@s{16.4} NextMinute \.{\defeq}}%
\@x{\@s{32.8} \.{\land} minute \.{\neq} 59}%
\@x{\@s{32.8} \.{\land} hour \.{'} \.{=} hour}%
\@x{\@s{32.8} \.{\land} minute \.{'} \.{=} minute \.{+} 1}%
\@x{\@s{16.4} Next \.{\defeq}}%
\@x{\@s{32.8} \.{\lor} NextMinute}%
\@x{\@s{32.8} \.{\lor} NextHour}%
\end{tlatex}
 \newline


Here's a breakdown of what the spec does:
\begin{itemize}
    \item \textit{Next} can take either \textit{NextHour} or \textit{NextMinute}.
    $\lor$ is the \textit{logical or} operator.
    \item \textit{Next} takes \textit{NextMinute} when \textit{minute} is not
    59. When \textit{NextMinute} takes: \textit{hour} in the next state equals
    to \textit{hour} in the current state, but \textit{minute} in the next state 
    is \textit{minute} in current state plus one.
    \item \textit{Next} takes \textit{NextHour} when \textit{minute} is 59.
    When \textit{NextMinute} takes: \textit{hour} in the next state equals to
    \textit{hour} in the current state plus one and modulus by number of hours in
    a day and \textit{minute} in the next state equals to zero. 
\end{itemize}

Note that the formulas are \textit{state descriptions}, not \textit{assignment}.
\textit{minute = 59} describes the state transition takes when minute
\textit{equals} 59. Since this is an equality description, \textit{minute = 59}
and \textit{59 = minute} are equivalent in TLA+.\newline

Finally, the Spec itself is formally defined as:\newline
\begin{tla}
    Spec ==
        /\ Init
        /\ [][Next]_vars
\end{tla}
\begin{tlatex}
\@x{\@s{16.4} vars\@s{0.63} \.{\defeq} {\langle} hour ,\, minute {\rangle}}%
\@x{\@s{16.4} Spec \.{\defeq}}%
\@x{\@s{32.8} \.{\land} Init}%
\@x{\@s{32.8} \.{\land} {\Box} [ Next ]_{ vars}}%
\end{tlatex}
\newline

Note this forms the basis for \textbf{all} TLA+ spec. Every example in this book
will include a \textit{Spec} definition similar to this.\\

$\Box[Next]_{vars}$ deserves some special attention:
\begin{itemize}
    \item $vars$ is defined earlier to be \textit{all} variables in the spec, in
    this case, hour and minute. A combination of these variables at different
    values constitutes the states of the system (eg. 23:59 and 00:00 are different states in the system).
    \item $\Box[Next]_{vars}$ is a box-action formula, where \textit{Next} is an
    action and \textit{vars} is a state function.
    \item $\Box$ (necessity operator) asserts the formula is always true for every step in the behavior.
    \item Steps in the behaviour are defined as $[Next]_{vars}$, where $Next$
    describes the action and $vars$ capturing all variables representing the state.
\end{itemize}

This can be roughly translated to: the system is valid for every step
\textit{Next} can take.

\section{Safety}

A safety property describes an invariant that must hold in \textit{every} state
of the system. A common invariant is \textit{type safety} checks. In a digital
clock, an hour can only be in value between 0 to 23, and a minute can only be the value
of 0 to 59:\newline

\begin{tla}
    Type_OK == 
        /\ hour \in 0..23
        /\ minute \in 0..59
\end{tla}
\begin{tlatex}
\@x{\@s{16.4} Type\_OK \.{\defeq}}%
\@x{\@s{32.8} \.{\land} hour \.{\in} 0 \.{\dotdot} 23}%
\@x{\@s{32.8} \.{\land} minute \.{\in} 0 \.{\dotdot} 59}%
\end{tlatex}
\newline

When an hour or minute falls outside of the specified range, the model checker 
reports violation.

\section{Liveness and Fairness}

Liveness property verifies certain behaviors across a sequence of states. One
liveness property is to confirm the clock wraps around at midnight, a property
that can only be verified after checking at least two states: \newline

\begin{tla}
    Liveness ==
        /\ hour = 23 /\ minute = 59 ~> hour = 0 /\ minute = 0
\end{tla}
\begin{tlatex}
\@x{\@s{16.4} Liveness \.{\defeq}}%
 \@x{\@s{32.8} \.{\land} hour \.{=} 23 \.{\land} minute \.{=} 59 \.{\leadsto}
 hour \.{=} 0 \.{\land} minute \.{=} 0}%
\end{tlatex}
\newline

$\leadsto$ is the \textit{leads to} operator, suggesting something is eventually
true. TLA+ provides a set of operators to describe the liveness property.\newline 

To verify liveness, we need to modify the spec slightly to enable
\textit{fairness} to prevent \textit{stuttering}. In plain terms, fairness
ensures a state always transitions to \textit{some other state}. Without fairness, the spec is allowed to \textit{stutter}, or \textit{not transition} to any state. This fails the liveness property check as the model checker cannot verify the behavior across a sequence of states. To get a more 
comprehensive description of fairness, refer to the last part of the
book.\newline

\begin{tla}
    Spec ==
        /\ Init
        /\ [][Next]_vars
        /\ WF_vars(Next)
\end{tla}
\begin{tlatex}
\@x{\@s{16.4} Spec \.{\defeq}}%
\@x{\@s{32.8} \.{\land} Init}%
\@x{\@s{32.8} \.{\land} {\Box} [ Next ]_{ vars}}%
\@x{\@s{32.8} \.{\land} {\WF}_{ vars} ( Next )}%
\end{tlatex}
\newline

$WF_{vars}(Next)$ is the fairness qualifier.

% TODO: insert reference here to specify systems 8.1 

\section{Model Checker}

A TLA+ spec can be verified using a model checker. The model checker runs the spec
and verifies all specified safety and liveness properties are fulfilled. The
model checker is a library written in Java and can be invoked from the command
line. For instructions on installing the model checker and related tools, please
see \cite{toolbox}.\newline

After installing the model checker, we need two things to verify the spec:
\begin{itemize}
    \item clock.tla: spec 
    \item clock.cfg: config file
\end{itemize}

For reference, clock.tla is listed below:\newline

\begin{tla}
--------------------------- MODULE clock ----------------------------
EXTENDS Naturals
VARIABLES hour, minute
vars == <<hour, minute>>
Type_OK == 
    /\ hour \in 0..23
    /\ minute \in 0..59
Liveness ==
    /\ hour = 23 /\ minute = 59 ~> hour = 0 /\ minute = 0
Init ==
    /\ hour = 0
    /\ minute = 0
NextMinute ==
    /\ minute = 59 
    /\ hour' = (hour + 1) % 24
    /\ minute' = 0
NextHour == 
    /\ minute # 59
    /\ hour' = hour 
    /\ minute' = minute + 1 
Next ==
    \/ NextMinute
    \/ NextHour
Spec ==
  /\ Init
  /\ [][Next]_vars
  /\ WF_vars(Next)
=============================================================================
\end{tla}
\begin{tlatex}
\@x{}\moduleLeftDash\@xx{ {\MODULE} clock}\moduleRightDash\@xx{}%
\@x{ {\EXTENDS} Naturals}%
\@x{ {\VARIABLES} hour ,\, minute}%
\@x{ vars \.{\defeq} {\langle} hour ,\, minute {\rangle}}%
\@x{ Type\_OK \.{\defeq}}%
\@x{\@s{16.4} \.{\land} hour \.{\in} 0 \.{\dotdot} 23}%
\@x{\@s{16.4} \.{\land} minute \.{\in} 0 \.{\dotdot} 59}%
\@x{ Liveness \.{\defeq}}%
 \@x{\@s{16.4} \.{\land} hour \.{=} 23 \.{\land} minute \.{=} 59 \.{\leadsto}
 hour \.{=} 0 \.{\land} minute \.{=} 0}%
\@x{ Init \.{\defeq}}%
\@x{\@s{16.4} \.{\land} hour \.{=} 0}%
\@x{\@s{16.4} \.{\land} minute \.{=} 0}%
\@x{ NextMinute \.{\defeq}}%
\@x{\@s{16.4} \.{\land} minute \.{=} 59}%
\@x{\@s{16.4} \.{\land} hour \.{'} \.{=} ( hour \.{+} 1 ) \.{\%} 24}%
\@x{\@s{16.4} \.{\land} minute \.{'} \.{=} 0}%
\@x{ NextHour \.{\defeq}}%
\@x{\@s{16.4} \.{\land} minute \.{\neq} 59}%
\@x{\@s{16.4} \.{\land} hour \.{'} \.{=} hour}%
\@x{\@s{16.4} \.{\land} minute \.{'} \.{=} minute \.{+} 1}%
\@x{ Next \.{\defeq}}%
\@x{\@s{16.4} \.{\lor} NextMinute}%
\@x{\@s{16.4} \.{\lor} NextHour}%
\@x{ Spec \.{\defeq}}%
\@x{\@s{8.2} \.{\land}\@s{0.16} Init}%
\@x{\@s{8.2} \.{\land}\@s{0.16} {\Box} [ Next ]_{ vars}}%
\@x{\@s{8.2} \.{\land}\@s{0.16} {\WF}_{ vars} ( Next )}%
\@x{}\bottombar\@xx{}%
\end{tlatex}

The corresponding clock.cfg is listed below: 
\begin{framed}
% \colorlet{shadecolor}{LavenderBlush2}
\begin{verbatim}
SPECIFICATION Spec
INVARIANTS Type_OK
PROPERTIES Liveness
\end{verbatim}
\end{framed}

After putting both clock.cfg and clock.tla in the same directory, one can now
run the model checker. In this book I'll assume a command line interface for the
model checker:
\begin{verbatim}
java -cp /usr/local/lib/tla2tools.jar tlc2.TLC clock
...
Model checking completed. No error has been found.
...
The depth of the complete state graph search is 1440.
\end{verbatim}
The 1440 states in the state graph represent the total number of minutes in a day.
