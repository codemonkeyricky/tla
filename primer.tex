\chapter{TLA+ Primer}

\section{Purpose}

The key insight to TLA+ is modeling a system as a state machine. A simple
digital clock can be represented by two variables, hour and minute and the
number of possible states in a digital clock is $24 * 60 = 1440$. For example,
a clock in state 10:00 will transition to state 10:01. Asssume an arbitrarily
system described by N variables, each variable having K possible values such
arbitrary system can have up to $N^K$ state.\newline

For every specification, designer can specify \textit{safety} proerty (or
invariants) that must be true in \textit{every} states. For example, in any
state of the digital clock hour \textit{must} be between 0 to 23, or formally
described as $hour \in 0..23$. Similarly, minute must have value between 0 to
59, or $minute \in 0..59$. Examples invariants of a system include: Only one
thread has exclusive access to a critical region, all variables in the system
are within allowable value, resource allocation manager never allocates more
than available resources.\newline

Designer can also specify \textit{liveness} property. These are properties to be
satisfied by a \textit{sequence of state}. One liveness property for the digital
clock could be when the clock is 10:00, it will eventually become 11:00 (10:00
\textit{leads} to 11:00). Example liveness property include: a distributed
system eventually converges, the scheduler eventually schedules every tasks in
the task queue, the resource allocation manager fairly allocates resources.
\newline

A TLA+ Spec can be checked by TLC, the model checker. TLC uses
\textit{breath-first search} algorithm to explore \textit{all} states in the
state machine and ensure safety and liveness properties are upheld.\newline

A TLA+ Spec describes the system using \textit{temporal logic}. The syntax may 
appear unfamiliar if one hasn't seen it before, but like any other programming 
language an initiated reader should become familiarized quickly. In this book I
will use 

\section{Design}

In this example, we will specify a \textit{digital clock}. The digital clock has
a few simple requirements:
\begin{itemize}
    \item Two variables to represent state: hour and minute
    \item The clock increment one minute at a time
    \item Hour is between 0 to 23, inclusive
    \item Minute is between 0 to 59, inclusive
    \item Clock wraps around at midnight (ie. 23:59 transitions to 00:00)
\end{itemize}

\section{Spec}

The \textit{Init} state of such system can be described as: \newline
\begin{tla}
    vars == <<hour, minute>>
    Init ==
        /\ hour = 0
        /\ minute = 0
\end{tla}
\begin{tlatex}
\@x{\@s{16.4} Init \.{\defeq}}%
\@x{\@s{32.8} \.{\land} hour \.{=} 0}%
\@x{\@s{32.8} \.{\land} minute \.{=} 0}%
\end{tlatex}
 \newline

$\defeq$ is the \textit{defines equal} symbol and $\land$ is the \textit{logical
and} symbol. The above TLA+ syntax can be read as \textit{Init} state is defined
as both hour and minute are both 0.\newline

The spec also always include a \textit{Next} definition, an \textit{action
formula} describing how the system transition from one state to another. Action
formula contains \textit{primed} variables what happens to the variable in its
next state. The \textit{Next} action for the digital clock can be defined
as:\newline

\begin{tla}
    NextHour ==
        /\ minute = 59 
        /\ hour' = (hour + 1) % 24
        /\ minute' = 0
    NextMinute == 
        /\ minute # 59
        /\ hour' = hour 
        /\ minute' = minute + 1 
    Next ==
        \/ NextMinute
        \/ NextHour
\end{tla}
\begin{tlatex}
\@x{\@s{16.4} NextHour \.{\defeq}}%
\@x{\@s{32.8} \.{\land} minute \.{=} 59}%
\@x{\@s{32.8} \.{\land} hour \.{'} \.{=} ( hour \.{+} 1 ) \.{\%} 24}%
\@x{\@s{32.8} \.{\land} minute \.{'} \.{=} 0}%
\@x{\@s{16.4} NextMinute \.{\defeq}}%
\@x{\@s{32.8} \.{\land} minute \.{\neq} 59}%
\@x{\@s{32.8} \.{\land} hour \.{'} \.{=} hour}%
\@x{\@s{32.8} \.{\land} minute \.{'} \.{=} minute \.{+} 1}%
\@x{\@s{16.4} Next \.{\defeq}}%
\@x{\@s{32.8} \.{\lor} NextMinute}%
\@x{\@s{32.8} \.{\lor} NextHour}%
\end{tlatex}
 \newline

Here's a breakdown of what the spec does:
\begin{itemize}
    \item \textit{Next} can take either \textit{NextHour} or \textit{NextMinute}
    \item \textit{Next} takes \textit{NextMinute} when \textit{minute} is not
    59. \textit{NextMinute} doesn't update \textit{Hour} and increments \textit{Minute}.
    \item \textit{Next} takes \textit{NextHour} when \textit{minute} is 59.
    \textit{NextHour} increments \textit{hour} modulus 24 and sets \textit{minute} to 0.
\end{itemize}

Note that the formulas are \textit{state descriptions}, not \textit{assignment}.
\textit{minute = 59} describes the state transition takes when minute
\textit{equals} 59. Since this is an equality description, \textit{minute = 59}
and \textit{59 = minute} are equivalent in TLA+.\newline

Finally, the Spec itself is formally defined as:\newline
\begin{tla}
    Spec ==
        /\ Init
        /\ [][Next]_vars
\end{tla}
\begin{tlatex}
\@x{\@s{16.4} vars\@s{0.63} \.{\defeq} {\langle} hour ,\, minute {\rangle}}%
\@x{\@s{16.4} Spec \.{\defeq}}%
\@x{\@s{32.8} \.{\land} Init}%
\@x{\@s{32.8} \.{\land} {\Box} [ Next ]_{ vars}}%
\end{tlatex}
\newline

Note this forms the basis for \textbf{all} TLA+ Spec. Every example in this book
will include a Spec definition similar to this. \newline

$\Box[Next]_{vars}$ deserves some special attention:
\begin{itemize}
    \item $vars$ is defined earlier to be \textit{all} variables in the spec, in
    this case hour and minute. Combination of these variables at different
    values constitute the states of the system (eg. 23:59 and 00:00 are differente
    states in the system).
    \item $\Box[Next]_{vars}$ is a box-action formula, where \textit{Next} is an
    action and \textit{vars} is a state function.
    \item $\Box$ operator asserts the formula is always true for every step in the behaviour.
    \item And steps in the behaviour is defined as $[Next]_{vars}$, where $Next$
    describe the action and $vars$ capturing all variables representing the state.
\end{itemize}

This can be roughly translate to: the system is valid for for every step
\textit{Next} can take, forming the basis of the Spec.

\section{Safety}

Safety property describes invariant that must hold true in \textit{every} state
of system. A common invariant is \textit{type safety} checks. In a digital
clock, hour can only be in value between 0 to 23, and minute can only be value
of 0 to 59:\newline

\begin{tla}
    Type_OK == 
        /\ hour \in 0..23
        /\ minute \in 0..59
\end{tla}
\begin{tlatex}
\@x{\@s{16.4} Type\_OK \.{\defeq}}%
\@x{\@s{32.8} \.{\land} hour \.{\in} 0 \.{\dotdot} 23}%
\@x{\@s{32.8} \.{\land} minute \.{\in} 0 \.{\dotdot} 59}%
\end{tlatex}
\newline

When hour or minute falls outside of the specified range, the model checker 
reports failure.

\section{Liveness}

Liveness property verifies certain behavioural across a sequence of state. One
liveness property is to confirm the clock wraps around at midnight, a property
that can only be verified after checking at least two states: \newline

\begin{tla}
    Liveness ==
        /\ hour = 23 /\ minute = 59 ~> hour = 0 /\ minute = 0
\end{tla}
\begin{tlatex}
\@x{\@s{16.4} Liveness \.{\defeq}}%
 \@x{\@s{32.8} \.{\land} hour \.{=} 23 \.{\land} minute \.{=} 59 \.{\leadsto}
 hour \.{=} 0 \.{\land} minute \.{=} 0}%
\end{tlatex}
\newline

$\leadsto$ is the \textit{leads to} operator, suggesting something is eventually
true. TLA+ provides a set of operators to describe liveness property.\newline 

To verify liveness, we need to modify the spec slightly to enable
\textit{fairness} to prevent \textit{stuttering}. In plain terms, fairness
ensure a state always transition to \textit{some other state}. Without fairness
the spec is allowed to \textit{stutter}, or \textit{not transition} to any state. 
This by definition fails liveness property check as the model checker is
unable to verify the behaviour across a sequence of states. To get a more 
comprehensive description of fairness, refer to the last part of the
book.\newline

\begin{tla}
    Spec ==
        /\ Init
        /\ [][Next]_vars
        /\ WF_vars(Next)
\end{tla}
\begin{tlatex}
\@x{\@s{16.4} Spec \.{\defeq}}%
\@x{\@s{32.8} \.{\land} Init}%
\@x{\@s{32.8} \.{\land} {\Box} [ Next ]_{ vars}}%
\@x{\@s{32.8} \.{\land} {\WF}_{ vars} ( Next )}%
\end{tlatex}
\newline

$WF_{vars}(Next)$ is the fairness qualifier.

% TODO: insert reference here to specifying systems 8.1 

\section{Model Checker}

The TLA+ spec can be verified using TLC model checker. The model checker runs
the spec and verifies all configured safety and liveness properties are
satisfied. In short, the model checker a library written in Java, and can be
invoked using command line java. For instruction on installing the model checker
and related tools, please see \cite{toolbox}.\newline

After installing the model checker, we need two things to verify the spec:
\begin{itemize}
    \item clock.tla: spec 
    \item clock.cfg: config file
\end{itemize}

For reference, clock.tla is listed below:\newline

\begin{tla}
--------------------------- MODULE clock ----------------------------
EXTENDS Naturals
VARIABLES hour, minute
vars == <<hour, minute>>
Type_OK == 
    /\ hour \in 0..23
    /\ minute \in 0..59
Liveness ==
    /\ hour = 23 /\ minute = 59 ~> hour = 0 /\ minute = 0
Init ==
    /\ hour = 0
    /\ minute = 0
NextMinute ==
    /\ minute = 59 
    /\ hour' = (hour + 1) % 24
    /\ minute' = 0
NextHour == 
    /\ minute # 59
    /\ hour' = hour 
    /\ minute' = minute + 1 
Next ==
    \/ NextMinute
    \/ NextHour
Spec ==
  /\ Init
  /\ [][Next]_vars
  /\ WF_vars(Next)
=============================================================================
\end{tla}
\begin{tlatex}
\@x{}\moduleLeftDash\@xx{ {\MODULE} clock}\moduleRightDash\@xx{}%
\@x{ {\EXTENDS} Naturals}%
\@x{ {\VARIABLES} hour ,\, minute}%
\@x{ vars \.{\defeq} {\langle} hour ,\, minute {\rangle}}%
\@x{ Type\_OK \.{\defeq}}%
\@x{\@s{16.4} \.{\land} hour \.{\in} 0 \.{\dotdot} 23}%
\@x{\@s{16.4} \.{\land} minute \.{\in} 0 \.{\dotdot} 59}%
\@x{ Liveness \.{\defeq}}%
 \@x{\@s{16.4} \.{\land} hour \.{=} 23 \.{\land} minute \.{=} 59 \.{\leadsto}
 hour \.{=} 0 \.{\land} minute \.{=} 0}%
\@x{ Init \.{\defeq}}%
\@x{\@s{16.4} \.{\land} hour \.{=} 0}%
\@x{\@s{16.4} \.{\land} minute \.{=} 0}%
\@x{ NextMinute \.{\defeq}}%
\@x{\@s{16.4} \.{\land} minute \.{=} 59}%
\@x{\@s{16.4} \.{\land} hour \.{'} \.{=} ( hour \.{+} 1 ) \.{\%} 24}%
\@x{\@s{16.4} \.{\land} minute \.{'} \.{=} 0}%
\@x{ NextHour \.{\defeq}}%
\@x{\@s{16.4} \.{\land} minute \.{\neq} 59}%
\@x{\@s{16.4} \.{\land} hour \.{'} \.{=} hour}%
\@x{\@s{16.4} \.{\land} minute \.{'} \.{=} minute \.{+} 1}%
\@x{ Next \.{\defeq}}%
\@x{\@s{16.4} \.{\lor} NextMinute}%
\@x{\@s{16.4} \.{\lor} NextHour}%
\@x{ Spec \.{\defeq}}%
\@x{\@s{8.2} \.{\land}\@s{0.16} Init}%
\@x{\@s{8.2} \.{\land}\@s{0.16} {\Box} [ Next ]_{ vars}}%
\@x{\@s{8.2} \.{\land}\@s{0.16} {\WF}_{ vars} ( Next )}%
\@x{}\bottombar\@xx{}%
\end{tlatex}

The corresponding clock.cfg is listed below: 
\begin{framed}
% \colorlet{shadecolor}{LavenderBlush2}
\begin{verbatim}
SPECIFICATION Spec
INVARIANTS Type_OK
PROPERTIES Liveness
\end{verbatim}
\end{framed}

After putting both clock.cfg and clock.tla in the same directory, one can now
run the model checker. In this book I'll assume a commandline interface for the
model checker:
\begin{framed}
\begin{verbatim}
java -cp /usr/local/lib/tla2tools.jar tlc2.TLC clock
... 
Model checking completed. No error has been found.
...
The depth of the complete state graph search is 1440.
\end{verbatim}
\end{framed}
The 1440 states in the graph represents total number of minutes in a day.
