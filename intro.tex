\chapter{Motivation}

\section{Catching Problems Early} 

Years ago, I worked on a proprietary low-power processor in an embedded
system. The processor ran a microcode featuring a custom instruction set. To
enter a low-power state, a set (possibly hundreds) of instructions were
executed. These instructions progressively put the system in a lower power state.
For example: Turn off IP A, then turn off IP B, then turn off the power island
to the IPs. To save cost and power, the low-power processor had very limited
debuggability support.\\

An experienced reader may start to notice some red flags.\\

If the microcode attempts to access the memory interface when the power
island has been shut off, the processor will hang. Since the debug power island has
been shut off, the physical hardware debug port is also unavailable, leaving the
developer with \textit{no way} of live debugging problems. At this point,
the developer needs to search through numerous instructions to
catch system constraint violations (invariants) \textit{manually}.\\

As one can imagine, maintaining the microcode was very expensive.
Fortunately, the proprietary low-power processor only had a handful of
instructions, so I created a simulator for this proprietary processor to verify
the microcode before deploying it on-target. The simulator models the processor
states as a state graph, with executed instruction, transitions the state machine
to the next state. At every state, all the invariants are verified. Example
invariants include:
\begin{itemize} 
    \item Accessing memory interface after power off leads to a hang 
    \item Accessing certain register in certain chip revision leads to a hang 
    \item Verify IPs are shut off in the allowed order
\end{itemize}
The verification algorithm was implemented using a \textit{depth-first-search},
providing 100\% microcode coverage before deployment on target.\\

To generalize, we can model an arbitrary system as a set of states and a set of invariants that must be upheld at all times. The complexity of such an arbitrary system generally grows quadratically as the number of states grows linearly (eg.
In an N-state system, adding state N+1 may introduce N transitions into the new
state). There are many engineering problems with a large number of states, such
as lockless or wait-free data structures, distributed algorithms, OS schedulers,
consensus protocols, and more. As the number of states grows, the problem becomes
more challenging for designers to reason about.\\

So, how do we produce a system that is \textit{correct by design?} 

\section{The Generalized Problem}

Fast forward to now: I stumbled across TLA+, a formalized solution of what I
was looking for.\\

The Turing Award winner Leslie Lamport invented the TLA+ in 1999, but TLA+
didn't appear to have caught on until the 2010s. My opinion is that TLA+ was
invented ahead of its time, and the problem complexity finally caught up in the
past decade or so, to allow TLA+ to demonstrate its strength.\newline

We are at a point in the technology curve where vertical scaling is no
longer practical, with CPU speed plateaued in the past decade or so. The industry
is exploring horizontal scaling solutions, such as hardware vendors focusing on
adding more CPU cores, or software vendors buying more low-end hardware instead
of a less high-end hardware. This shifts the technology complexity from vertical
to horizontal, demanding solutions to maximize concurrent resource
utilization. There is one slight problem though:\\

\textit{Humans are not good at concurrent reasoning}.\\

Our cognitive system is optimized for sequential reasoning. Enumerating
all scenarios in one's mind to ensure an arbitrary design accommodates all
the corner cases is challenging.\\

Consider a distributed system. The system is a cluster of independently operating
entities, which need to somehow collectively offer the correct system behavior,
while any one of the machines may receive instructions out of order, crash,
recover, etc.\\

Consider a single producer multiple consumer lockless queue. The consumers may
reserve an index in the queue in a certain order but may release it in a
different order. What if one reader is slow, and another reader is superfast
and possibly lapses the slow reader?\\

Consider an OS scheduler with locks. Assume all the processes have the same
priority. Can a process starve the other processes by repeatedly acquiring and
releasing the lock? How do we ensure scheduling is fair?\\

The \textit{anti-pattern} is to keep band-aiding the design until the user stops
filing bug reports. This is never ideal. Per Murphy's law, anything that can go
wrong \textit{will go wrong}, and a hard-to-reproduce bug will come in at the
most inconvenient time. How do we make sure the solution is \textit{correct by
design}? To solve this problem, we must rely on tools to do the reasoning
\textit{for us}.

\section{What is TLA+?}

TLA+ is a \textit{system specification language} to describe a system
without implementation details. TLA+ allows a designer to describe a system as a
set of states with transitions from one state to the next. Designers can
describe invariants that must hold in every state and liveness properties a
sequence of states must satisfy. One of TLA+'s keys is once the system is modeled
as a finite set of states, the states can be \textit{exhaustively} explored
(via breath-first-search) to ensure properties are upheld throughout the entire
state space (either per state or a sequence of states).

\section{About This Book}
To my surprise, there is not as much material on TLA+ as I assumed for
such a critical tool in a designer's toolbox. This book was initially a set of
notes I took while learning TLA+. I decided to formalize these notes into this
short book, which I hope the readers will find helpful in their TLA+
journey.\newline

The book intends to teach the reader how to write TLA+ spec for their design to
provide confidence in \textit{design correctness}. This book is targeted at
software designers, hardware designers, system architects, and in general anyone
interested in designing correct systems.\newline 

To get the most out of the book, the reader should have general computer
science knowledge. The reader doesn't need to be an expert in a particular
language to understand this book; TLA+ is effectively its language. This book is
example-driven and will go through designs such as lockless queues, simple
task schedulers, consensus algorithms, etc. Readers will likely enjoy a deeper
insight if there is familiarity with these topics.

\section{How to Use This Book}

This book was designed to be used as a reference, providing examples
and references using TLA+.\newline

Examples are split into two categories: Examples written using TLA+ and
examples written using PlusCal (the C-like syntax that transpiles down to TLA+).
I believe they are useful for different use cases. The differences will
be highlighted in their respective sections. All examples will follow a
similar layout, covering the problem statement, design, spec, and safety
properties.\newline

All examples in this book will be presented using TLA+ \textit{mathematical
notation}. Converting between Mathematical and ASCII notation is trivial due to
the one-to-one mapping. Readers are encouraged to consult Table 8 in \cite{ss}
as needed.\\

The last part of the book provides language references and some focused topics.
Readers can use the last part of the book as a general reference. 
