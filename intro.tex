\chapter{Introduction}

\section{Catching Problems Early}

Years ago, I worked on a propietary low power processor in an embedded system.
The processor ran microcode featuring a custom instruction set. To enter a low
power state, a set of (possibly hundreds) instructions were executed. These
instructions progressively puts the system in lower power state. For example:
turn off IP A, then turn off IP B, then turn off the power island to the IPs. To
save cost and power, the low power processor had very limited debuggability
support.\newline

An experienced reader may start to notice some redflags.\newline

If the microcode attempts to access the memory interface when the power island
has been shut off, processor would hang. Since the power island has been shut
off, the physical hardware debug port is also unavailable, leaving the developer
with \textit{no way} of live debugging related problem. At this point the
developer needs to siphon through (possibly hundreds) of instructions to catch
invariant violation \textit{manually}.\newline

As one can imagine, maintaining the microcode was very expensive. Fortunately,
the propietary low power processor only had a handful of instructions, I created
an emulator for this propietary processor to verify the microcode prior to
deploying it on target. The emulator models the processor states as a state
graph, with executed instruction transitions the state machine to the next
state. At every state all the invariants are evaluated to ensure none have been
violated. Some of the invariants included:
\begin{itemize}
    \item Accessing memory interface after power off leads to a hang
    \item Accessing certain register in certain chip revision leads to a hang 
    \item Verify IPs are shut off in the allowed order
\end{itemize}

The verification algorithm was implemented using a \textit{depth-first search}
algorithm, providing \textit{100\%} microcode coverage before deploy on
target.\newline

To generalize, we can model arbitrary system as a set of states with a
collection of invariant that must be upheld at all times. The complexity of the
such arbitrarily system generally grows quadratically as the number of states
grow linearly (eg. in a N state system, adding state N+1 may introduce N
transitions into the new state). There are many engineering problems that
exhibit a large number of states, such as lockless or waitfree data structure,
distributed algorithms, OS scheduler, and more.\newline

\textit{So, how do we build a solution that is correct by design?}

\section{The Generalized Problem}

Fast forward to now: I stumbled across TLA+, a formalized solution of what I was
looking for.\newline

Leslie Lamport invented the TLA+ 1999, but TLA+ didn't appear to have caught on
until the 2010's. My personal opinion is TLA+ was invented ahead of its time,
and the problem complexity finally caught up in the past decade or so to allow
TLA+ to visibly demonstrate its strength.\newline

We are also at a point in the technology curve where vertical scaling is no
longer practical, with CPU speed plateau'd in the past decade or so. The
industry is exploring horizontal scaling solution, such as hardware vendor
focusing adding more CPU cores, or software vendors buying many low end hardware
instead of a few high end hardware. This shifts the technology complexity from
hardware to software, demanding software solution to maximize concurrent
hardware resource utilization. \newline

One slight problem: \textit{The cognitive load of of a person is between 5 to 9
items at most.}\newline

Humans are good at high level reasoning, but not so good at keeping track of
many things happening at the same time. It is hard to enumerate all possible
scenario in one's mind to ensure the design accommodates all the edge cases.
\newline

Consider a distributed system. The system is a cluster of independently
operating entities and need to somehow collectively offer the correct system
behaviour, while any one of the machines may receive instructions out of order,
crash, recover, etc. \newline

Consider a single producer multiple consumer lockless queue. The consumers may 
reserve an index in the queue in certain order, but may release them in different order. 
What if one reader is really slow, and another reader is super fast and possibly 
lapse the slow reader? \newline

Consider an OS scheduler with locks. Assume all the processes have the same
priority. Can a process starve the other processes by repeatedly acquire and
release the lock? How do we ensure scheduling is fair?\newline

The usual \textit{anti-pattern} is to keep bandaiding the design until bugs stop
comming. This is never ideal. Per Murphy's law, anything that can go wrong will
go wrong, and a hard to reproduce bug will come in at the most inconvinient
time. How do we make sure the solution is actually \textit{correct by design}?
To solve this problem, we must rely on tools to do the reasoning \textit{for
us}.

\section{TLA+}

TLA+ is a \textit{system specification language}, with the intent to describe
the system with implementation details removed. TLA+ allows designer to describe
the system as a sequence of states. The designer can expresses transition
condition from one state to another, describe invariants that must hold true in
every state and liveness properties that the overall system should converge to.
The key innovation of TLA+ is once the system is modeled as a finite state
machine, the states can be \textit{exhaustively} explored (via
breath-first-search) to ensure certain properties are held through out the
entire state space (either per state or a sequence of states).\newline

\section{About This Book}

This book was initially a set of notes I took as I was learning TLA+. To my
surprise, there's not as much material on TLA+ as I had assumed for such a
critical tool in the designer toolbox. I decided to formalize the notes into
this short book, which I hope the readers find this helpful in their journey to
master TLA+.  \newline

The intent for the book is to teach reader how to write TLA+ spec for their
design to provide confidence in \textit{design correctness}. The content to this
book is appropriate for software designer, hardware designer, system architect,
and such.\newline 

As for the readers: some computing science knowledge is required. One doesn't
need to be expert at a particular language to understand this book; TLA+ is
effectively its own language. This book is example driven and will go through
designs such as lockless queue, simple task scheduler, consensus algorithm, etc.
Reader will likely enjoy a deeper insight if she has some familiarity with these
topics.

\section{Book Layout}

This book was motivated by the intent to solve problems. The book is designed to
be example heavy with many chapter each represnting an problem that can be
modelled using TLA+.\newline

Examples are split into two categories: A set of examples written using native
TLA+ syntax, and another set of examples written using PlusCal (C-like syntax).
I believe they are useful under different use cases. The differences will be
highlighted in their respective sections. All examples will follow a similar 
layout, covering the expected design process (eg. requirement, spec, safety and
liveness properties). \newline

Finally, there will be part that language reference portion that that discuss a
few topics deserving extra attention. The intent is to be using this section of the 
book as a \textit{reference}.

