\chapter{General Guideline}

\section{Be Skeptical}

Even if model checker reports no violation, it is often worthwhile to dump and
audit the states to make sure the specification is defined correctly.

\begin{verbatim}
tlc elevator -dump out > /dev/null && cat out.dump | head -n5
State 1:
a = 1
State 2:
a = 2
\end{verbatim}

Designer can grep the output to look for the state being set to expected value
to confirm the specification is working as intended.

\section{Debug} 

Designer can insert prints or asserts with \textit{PrintT} and \textit{Assert}
respectively, these may help root causing a bug when model checker reports fails.\\

An \textit{Assert} dumps the backtrace leading up to it, but does not include values of
any non-state variables. For example, if the state includes with local macros
using \textit{LET..IN} semantics or \textit{RECURSIVE} function calls, the associated values will
not be displayed. Prints do not help in this case either. Since the model
checker explores the states using BFS, the print order is not guaranteed.\\

To display any intermediary values, designer can add an auxiliary debug variable
to the system definition. The debug variable will be dumped as part of the state
backtrace.

\section{Dead Lock}

Deadlock typically happens when the model checker runs out of things to do. This
can be a result of an incomplete specification definition, where certain edge
cases were not accounted for. The model checker typically provides a fairly
comprehensive backtrace leading up to the deadlock to simplify debugging.

\section{Safety Properties}

Designer should add as many system safety properties as possible to catch
problems early. Safety properties are checked at the end of every state and
impose very little overhead to overall execution.

\section{Live Lock}

Livelock happens when the model checker identifies a case where the system is 
oscillating between states without making progress towards the final goal.  An
example is the elevator stuck going between two floors instead of going to the
top floor, or the system is stuck dropping and retransmitting the same packet.\\

These can be fixed by providing additional fairness descriptions to the
specification, instructing the model checker how to prioritize transitions to
take.\\

For a detailed fairness description please refer to Chapter~\ref{chap:fairness}.

\section{Liveness Properties}

While verifying liveness properties is super useful, the model checker at the 
time of writing can only verify liveness properties sequentially. To verify
liveness, model checker must first identify the strongly connected component in
the graph, and the algorithm currently implemented does not support
parallelization.\\

The general recommendation is to only define and verify liveness properties if
the specification is sufficiently small. 

\section{Fully Verification}

Fully verified a model provide extremely high confidence the specification is
correct. However, since the specification complexity grows exponentially,
designer may need to get creative to reduce the model to allow the model checker can
run in a practical amount of time.\\

Since the model complexity grows exponentially, there's little value in
attempting to hyper-optimize implementation detail. Designers should focus on
simplifying the model by removing non-critical features and focus verifying
features with the highest return on investment.\\

One way of trimming out the low-value part of a specification is to audit the
state dump. Even in the case of a non-terminating run, a partial state dump may
help identify low-value abstractions that can be removed from the spec.\\

One key value of TLA+ is it highlights all the corner cases in the system. Even
if the designer ends up simplifying the specification, it still likely highlights certain
conditions the designer was previously unaware of.\\

% In general, when the specification has millions or higher more states, model checker 
% likely cannot verify it in a few seconds. If a fault is caught after verifying a
% million states, the model likely can be simplified to reproduce the fault in
% much less states.\\

\section{Simulation and Generation}

While fully verifying a specification is always ideal, it is not always
practical. With sufficient complexity, the model checker will not be able to 
verify the specification in reasonable time. That said, it is still worthwhile 
to verify the specification by running the model checker in either simulation or
generation mode. Simulation is the legacy simulation mode where the model
checker keeps running without termination. Generation mode also causes the model
checker to run without termination, but applies uniform probability distribution
on selecting next state transition.\\

Even for a complex specification, the majority of violation will be caught very
quickly per 80/20 rule. For some problem domain this may be good enough, and is
certainly better than no verification at all. 