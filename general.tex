\chapter{General Guideline}

\section{Debug}

Debugging in TLC is a bit different than debugging with normal programs. A step
in the model checker is a state transition. Even if the model checker completes,
it's still worthwhile to dump and audit the states just to make sure
\textit{Spec} is defined correctly.

\begin{verbatim}
tlc elevator -dump out > /dev/null && cat out.dump | head -n5
State 1:
a = 1
State 2:
a = 2
\end{verbatim}

You may want to grep the output to look for the state being set to expected
value to confirm your spec is working as intended.

\section{Dead Lock}

Deadlock typically happens when the model checker runs out of things to do. This
can be a result of an incomplete \textit{Spec} definition, where certain edge
cases were not accounted for. The model checker typically provides a fairly
comprehensive backtrace leading up to the deadlock to simplify debugging.

\section{Live Lock}

Livelock happens when the model checker identifies a case where the liveness
property is violated. An example is the elevator stuck going between two floors 
instead of going to the top floor, or the system is stuck dropping and
retransmitting the same packet.\\

These are typically fixed by providing additional fairness descriptions to the
Spec, telling the model checker how to continue in the case of a live lock.\newline 

For a detailed fairness description please refer to Chapter~\ref{chap:fairness}.

\section{Model Refinement}

This is the \textit{art} of enabling a TLA+ spec to be verifiable. Model
checking is only valuable if it can be verified within a reasonable amount of
time. Since the model complexity grows exponentially, there's little value in
attempting to hyper-optimize the details.  Designers should focus on simplifying
the model by removing non-critical features and focus on features with highest
return on investment.\\

One useful way of trimming out the low-value portion of a spec is to audit the
state dump. Even in the case of a non-terminating run, a partial state dump may
help identify low-value abstractions that can be removed from the spec.\newline

One key value of TLA+ is it highlights all the corner cases in the system. Even
if the designer ends up simplifying the spec, it still likely highlights certain
conditions the designer was previously unaware of.\newline

In general, when the spec has millions or higher more states, it likely cannot 
be verified within a few seconds. If a fault is caught after verifying a million
states, the model likely can be simplified to reproduce the fault in much less
states.
